\documentclass[mathserif
, handout
]{beamer}
 
 % \useoutertheme{wuerzburg}
%  \useinnertheme[outline]{chamfered}

\usepackage{tabu}
\usepackage{rotating}
\usepackage[]{algorithm2e}
\usepackage{color, colortbl}
%\usepackage{default}
\usepackage{fontspec}
\usepackage{polyglossia} 
\setmainlanguage{vietnamese}
%\setdefaultlanguage{vietnamese} 
%\setmainfont{Palatino}
 \usepackage{wasysym}
\usepackage{pifont}% http://ctan.org/pkg/pifont

\usepackage{multicol}
\usepackage{sidecap}

\usepackage{hyperref}

\usepackage{pgf}    
\usepackage{tikz}
\usetikzlibrary{arrows,automata,decorations.pathmorphing,backgrounds,positioning,fit}
\usepackage{array}
\usepackage{listings}

\usepackage{enumerate}
%\usepackage{amsmath,mathtools}
%		\usepackage{fink}

\usepackage{amsmath,amsthm, amssymb}
\usepackage{microtype}
\usetikzlibrary{arrows,automata}
\usetikzlibrary{decorations.pathmorphing}


\usetikzlibrary{calc}

 

\usetikzlibrary{trees}
\usepackage{listings}

\setbeamertemplate{footline}[frame number]
\setbeamertemplate{navigation symbols}{}%remove navigation symbols

%\usepackage{listingsutf8}

%\setbeameroption{show notes on second screen=right}


  
%\newtheorem{Lemme}{Bổ đề} 
%\newtheorem*{LI}{Lemme d'itération infinie}

%\newtheorem{Proposition}{Mệnh đề}[section]
%\newtheorem{Theorem}{Định lý}[section]
%\newtheorem{Corollaire}{Hệ quả}[section]
%\newtheorem*{Conjecture}{Giả thuyết}
% \newtheorem*{Probleme}{Bài toán}
% \newtheorem*{Fait}{Fait}
% 
% 
% \theoremstyle{definition} \newtheorem{Definition}{Định nghĩa}
% \theoremstyle{definition} \newtheorem{example}{Ví dụ}
% \theoremstyle{remark} \newtheorem*{Remarque}{Chú ý}


\usetikzlibrary{arrows,automata}



\usetikzlibrary{trees}


% \newcommand{\mnvect}[2]
% {
%   \begin{bmatrix}	#1\\#2
%   \end{bmatrix}
% }

\definecolor{olive}{rgb}{0.3, 0.4, .1}
\definecolor{fore}{RGB}{249,242,215}
\definecolor{back}{RGB}{51,51,51}
\definecolor{title}{RGB}{255,0,90}
\definecolor{dgreen}{rgb}{0.,0.7,0. }
\definecolor{gold}{rgb}{1.,0.84,0.}
\definecolor{JungleGreen}{cmyk}{0.99,0,0.52,0}
\definecolor{BlueGreen}{cmyk}{0.85,0,0.37,0}
\definecolor{RawSienna}{cmyk}{0,0.72,1,0.45}
\definecolor{Magenta}{cmyk}{0,1,0,0} 



%

%\setlength{\topmargin}{0cm} \setlength{\oddsidemargin}{0cm}
%\setlength{\evensidemargin}{0cm} \setlength{\textwidth}{17truecm}
%\setlength{\textheight}{21.0truecm}


%\parindent = 3 pt
%\parskip = 12 pt

%\newtheorem*{LI}{Lemme d'itération infinie}



\newtheorem{prprt}{Propriété}
\newtheorem{prpstn}{Mệnh đề}
\newtheorem{thrm}{Định lý}
\newtheorem{lmm}{Bổ đề}

\newtheorem{crllr}{Hệ quả}
\newtheorem{clm}{Fait}
\newtheorem{nt}{Notation}
 
\newtheorem*{cnjctr}{Conjecture}
\newtheorem{prblm}{Problème}
\newtheorem{qstn}{Question}
\newtheorem{fct}{Fait}
%\newtheorem{xmpl}{Exemple}
\newtheorem{rmrk}{Nhận xét}

\theoremstyle{example}
\newtheorem{xmpl}{Ví dụ}
\newtheorem{xrcs}{Bài tập}
  \newtheorem{dfntn}{Định nghĩa}
  

% \declaretheorem[name=Problème]{prblm}
% \declaretheorem[name=Question, style=remark, numbered=no]{qstn}

% \declaretheorem[name=Théorème, numberwithin=section]{thrm}
% \declaretheorem[name=Lemme, sibling=thrm]{lmm}
% \declaretheorem[ name=Propriété, sibling=thrm]{prprt}
% \declaretheorem[ name=Proposition, sibling=thrm]{prpstn}
% \declaretheorem[name=Corollaire, sibling=thrm]{crllr}
% \declaretheorem[name=Fait, sibling=thrm]{fct}
% \declaretheorem[name=Notation, sibling=thrm]{nt}


% \declaretheorem[style=definition, name=Définition, sibling=thrm]{dfntn}

% %\theoremstyle{definition} \newtheorem{dfntn}{Définition}[section]

% \renewcommand\thmcontinues[1]{reprise de p.\,\pageref{#1}}

% \declaretheorem[style=remark, name=Exemple%, numberwithin=section
% ]{xmpl}

% \declaretheorem[style=remark, name=Remarque, numbered=no]{rmrk}

% %\declaretheorem[style=definition,numberwithin=chapter,name = Exemple]{xmpl}

% %\theoremstyle{remark} \newtheorem{xmpl}{Exemple}[chapter]

% %\theoremstyle{remark} \newtheorem*{rmrk}{Remarque}



\newtheorem{cs}{Cas}


\def\mclose{\texttt{close}}
\def\mopen{\texttt{open}}

\def\mmclose{\texttt{\scriptsize close}}
\def\mmopen{\texttt{\scriptsize open}}



% \newcommand{\mvect}[2]
% {
% \bigl[ \begin{smallmatrix}
% #1\\ #2
% \end{smallmatrix} \bigr]
% }

% \newcommand{\mnvect}[2]
% {
%   \begin{bmatrix}	#1\\#2
%   \end{bmatrix}
% }

% % \newcommand{\mnvect}[2]
% % {
% % #1/#2
% %   % \begin{bmatrix}	#1\\#2
% %   % \end{bmatrix}
% % }

% \newcommand{\XMPL}[3]
% {
%   \begin{xmpl}
%     Soient $L=\{#1\}$ et $\Sigma=\{#2\}$. On peut vérifier que $L$ est \orl\ avec le
%     relateur de base $#3$.
%   \end{xmpl}
% }

% \newcommand{\XMP}[4]
% {
%   \begin{xmpl}[#4]
%     Soient $L=\{#1\}$ et $\Sigma=\{#2\}$. On peut vérifier que $L$ est \orl\ avec le
%     relateur de base $#3$.
%   \end{xmpl}
% }

% \newcommand{\Pui}[2]
% {
%   #1^{\leq #2}
% }


% % \newcommand{\XMPL1}[4]
% % {
% %   \begin{xmpl}
% %     Soient $L=\{#1\}$ et $\Sigma=\{#2\}$. Il est clair que $L$ est \orl\ avec le
% %     relateur de base $#3$. $L^\omega$ est un 
% %   \end{xmpl}
% % }

% \def\vvs{\vspace{11pt}}
% \def\nni{\noindent}


% \newcommand{\cas}[1]
% {
% \vvs\nni
% \textbf{Cas #1 :}
% }



% \newcommand{\souscas}[1]
% {
% \vvs\nni
% \textbf{Sous-cas #1 :}
% }

% \def\pcom{paire de mots incompatibles}
% \def\wpcom{paire de mots $\infty$-incompatible}

% \def\upcom{une paire de mots incompatibles}
% \def\uwpcom{une paire de mots $\infty$-incompatibles}
% \def\comp{\asymp}

% \def\wg{code générateur}

%  \def\gc{code générateur}

% \def\gcx{codes générateurs}
% \def\Gcx{Codes générateurs }
% \def\ugc{un code générateur}
% \def\Ugc{un Code générateur}

% \def\wgc{$\omega$-code générateur}
% \def\wgcx{$\omega$-codes générateurs}
% \def\wGcx{$\omega$-Codes générateurs }
% \def\wugc{un $\omega$-code générateur}
% \def\wUgc{un $\omega$-code générateur}

% \def\orl {langage à un relateur}

% \def\orlx {langages à un relateur}
% \def\Orlx {Langages à un relateur}
% \def\uorl {un langage à un relateur}


% \def\ugc{un code générateur}

% \def\cp{code préfixe}

% \def\iff{si et seulement si} 
% \def\w{\omega}

% \def\CODE{la proposition~\ref{c3prop23}, $L^\omega$ n'a pas de \gc}
% \def\NOCODE{$L^\omega$ n'a pas de \gc}


\def\vs{}
\def\ni{}





%\setlength{\topmargin}{0cm} \setlength{\oddsidemargin}{0cm}
%\setlength{\evensidemargin}{0cm} \setlength{\textwidth}{17truecm}
%\setlength{\textheight}{21.0truecm}


%\parindent = 3 pt
%\parskip = 12 pt

%\newtheorem*{LI}{Lemme d'itération infinie}



\newtheorem{prprt}{Propriété}
\newtheorem{prpstn}{Mệnh đề}
\newtheorem{thrm}{Định lý}
\newtheorem{lmm}{Bổ đề}
\newtheorem{rl}{Luật}

\newtheorem{crllr}{Hệ quả}
\newtheorem{clm}{Khẳng định}
\newtheorem{nt}{Notation}
 
\newtheorem*{cnjctr}{Giả thuyết}

\newtheorem{fct}{Fait}
%\newtheorem{xmpl}{Exemple}

\theoremstyle{example}
\newtheorem{xmpl}{Ví dụ}
\newtheorem{xrcs}{Bài tập}
  \newtheorem{dfntn}{Định nghĩa}
  \newtheorem{qstn}{Câu hỏi}
\newtheorem{prblm}{Bài toán}  
   \newtheorem{sol}{Lời giải}
\newtheorem{rmrk}{Nhận xét}
  
%  \newtheorem{rmrk}{Định nghĩa}
  

% \declaretheorem[name=Problème]{prblm}
% \declaretheorem[name=Question, style=remark, numbered=no]{qstn}

% \declaretheorem[name=Théorème, numberwithin=section]{thrm}
% \declaretheorem[name=Lemme, sibling=thrm]{lmm}
% \declaretheorem[ name=Propriété, sibling=thrm]{prprt}
% \declaretheorem[ name=Proposition, sibling=thrm]{prpstn}
% \declaretheorem[name=Corollaire, sibling=thrm]{crllr}
% \declaretheorem[name=Fait, sibling=thrm]{fct}
% \declaretheorem[name=Notation, sibling=thrm]{nt}


% \declaretheorem[style=definition, name=Définition, sibling=thrm]{dfntn}

% %\theoremstyle{definition} \newtheorem{dfntn}{Définition}[section]

% \renewcommand\thmcontinues[1]{reprise de p.\,\pageref{#1}}

% \declaretheorem[style=remark, name=Exemple%, numberwithin=section
% ]{xmpl}

% \declaretheorem[style=remark, name=Remarque, numbered=no]{rmrk}

% %\declaretheorem[style=definition,numberwithin=chapter,name = Exemple]{xmpl}

% %\theoremstyle{remark} \newtheorem{xmpl}{Exemple}[chapter]

% %\theoremstyle{remark} \newtheorem*{rmrk}{Remarque}



\newtheorem{cs}{Cas}


\def\mclose{\texttt{close}}
\def\mopen{\texttt{open}}

\def\mmclose{\texttt{\scriptsize close}}
\def\mmopen{\texttt{\scriptsize open}}



% \newcommand{\mvect}[2]
% {
% \bigl[ \begin{smallmatrix}
% #1\\ #2
% \end{smallmatrix} \bigr]
% }

% \newcommand{\mnvect}[2]
% {
%   \begin{bmatrix}	#1\\#2
%   \end{bmatrix}
% }

% % \newcommand{\mnvect}[2]
% % {
% % #1/#2
% %   % \begin{bmatrix}	#1\\#2
% %   % \end{bmatrix}
% % }

% \newcommand{\XMPL}[3]
% {
%   \begin{xmpl}
%     Soient $L=\{#1\}$ et $\Sigma=\{#2\}$. On peut vérifier que $L$ est \orl\ avec le
%     relateur de base $#3$.
%   \end{xmpl}
% }

% \newcommand{\XMP}[4]
% {
%   \begin{xmpl}[#4]
%     Soient $L=\{#1\}$ et $\Sigma=\{#2\}$. On peut vérifier que $L$ est \orl\ avec le
%     relateur de base $#3$.
%   \end{xmpl}
% }

% \newcommand{\Pui}[2]
% {
%   #1^{\leq #2}
% }


% % \newcommand{\XMPL1}[4]
% % {
% %   \begin{xmpl}
% %     Soient $L=\{#1\}$ et $\Sigma=\{#2\}$. Il est clair que $L$ est \orl\ avec le
% %     relateur de base $#3$. $L^\omega$ est un 
% %   \end{xmpl}
% % }

% \def\vvs{\vspace{11pt}}
% \def\nni{\noindent}


% \newcommand{\cas}[1]
% {
% \vvs\nni
% \textbf{Cas #1 :}
% }



% \newcommand{\souscas}[1]
% {
% \vvs\nni
% \textbf{Sous-cas #1 :}
% }

% \def\pcom{paire de mots incompatibles}
% \def\wpcom{paire de mots $\infty$-incompatible}

% \def\upcom{une paire de mots incompatibles}
% \def\uwpcom{une paire de mots $\infty$-incompatibles}
% \def\comp{\asymp}

% \def\wg{code générateur}

%  \def\gc{code générateur}

% \def\gcx{codes générateurs}
% \def\Gcx{Codes générateurs }
% \def\ugc{un code générateur}
% \def\Ugc{un Code générateur}

% \def\wgc{$\omega$-code générateur}
% \def\wgcx{$\omega$-codes générateurs}
% \def\wGcx{$\omega$-Codes générateurs }
% \def\wugc{un $\omega$-code générateur}
% \def\wUgc{un $\omega$-code générateur}

% \def\orl {langage à un relateur}

% \def\orlx {langages à un relateur}
% \def\Orlx {Langages à un relateur}
% \def\uorl {un langage à un relateur}


% \def\ugc{un code générateur}

% \def\cp{code préfixe}

% \def\iff{si et seulement si} 
% \def\w{\omega}

% \def\CODE{la proposition~\ref{c3prop23}, $L^\omega$ n'a pas de \gc}
% \def\NOCODE{$L^\omega$ n'a pas de \gc}


\def\vs{}
\def\ni{}


\def\trail{hành trình đơn}
\def\Trail{Hành trình đơn}

\def\ctrail{\trail\ đóng}
\def\Ctrail{\Trail\ đóng }

\def\walk{hành trình}
\def\Walk{Hành trình}

\def\cwalk{hành trình đóng}
\def\Cwalk{Hành trình đóng}

\def\path{đường đi}
\def\Path{Đường đi}
 
\def\conn{liên thông}
\def\Conn{Liên thông}

\def\Comp{Thành phần liên thông}
\def\comp{thành phần liên thông}

\def\Cuted{Cạnh cắt}
\def\cuted{cạnh cắt}

\def\Cutve{Đỉnh cắt}
\def\cutve{đỉnh cắt}

\def\Induced{Đồ thị con cảm sinh}
\def\induced{đồ thị con cảm sinh}

 
\def\iff{{\color{blue} nếu và chỉ nếu}}

\def\ideg{\text{indeg}}
\def\odeg{\text{outdeg}}

\def\pr{\mathrm{Pr}}
\def\ex{\mathrm{Ex}}
\def\S{\mathcal{S}}
\def\var{\mathrm{Var}}

\def\F{\mathbb{F}}
\def\Z{\mathbb{Z}}
\def\N{\mathbb{N}}
\def\ord{\mathrm{ord}}
\newcommand{\bigO}{\ensuremath{\mathcal{O}}}% big-O notation/symbol


 \newcommand{\defi}[1]{{\color{blue}{\textbf{\emph{#1}}}}}
\newcommand{\contradiction}{{\hbox{%
    \setbox0=\hbox{$\mkern-3mu\times\mkern-3mu$}%
    \setbox1=\hbox to0pt{\hss$\times$\hss}%
    \copy0\raisebox{0.5\wd0}{\copy1}\raisebox{-0.5\wd0}{\box1}\box0
}}}

\newcommand{\cmark}{{\color{blue}\Large\ding{51}}}%
\newcommand{\xmark}{{\color{red}\Large\ding{55}}}%

%\newcommand{\defi}[1]{{\color{blue}{\textbf{\emph{#1}}}}}


 \AtBeginSection[]  
 { 
   \begin{frame}[plain]{Nội dung} 
     \tableofcontents[currentsection,currentsubsection] 
   \end{frame} 
 }  



\begin{document}
% \tikzstyle{every picture}+=[remember picture]

% \tikzstyle{na} = [baseline=-.5ex]

\author{Trần Vĩnh Đức}
%\institute[HUST]{Hanoi University of Science and Technology}


% \newcommand{\cmark}{{\color{blue}\Large\ding{51}}}%
% \newcommand{\xmark}{{\color{red}\Large\ding{55}}}%
\title{Nhập môn số học thuật toán} 
 \author{Toán Chuyên Đề}   
\institute[HUST]{HUST}
    
\maketitle    

\begin{frame}{Tài liệu tham khảo}
  \begin{itemize}
  \item J. Hoffstein, J. Pipher, J. H. Silverman,
    \textit{An Introduction to Mathematical Cryptography},
    Springer-Verlag – Undergraduate Texts in Mathematics, 2nd Ed.,
    2014.
  \item T. H. Cormen, C. E. Leiserson, R. L. Rivest, C. Stein.  \textit{Introduction to Algorithms}, Third Edition (3rd ed.). The MIT Press. 2009.

  \item H. H. Khoái, \textit{Nhập môn số học thuật toán}
  \end{itemize}
%  \item Phan .Đ. Diệu, \textit{Logic toán \& cơ sở toán học}. (2003)  
\end{frame}

\section{Thuật toán Euclid}
\begin{frame}{Ký hiệu}
	\begin{itemize}
		\item $\mathbb{N} = \{1,2,3, \dots\}$
		\item $\mathbb{Z} = \{\dots,-2,-1,0,1, 2, \dots\}$
	\end{itemize}
\end{frame}

\begin{frame}%{Tính chia hết}
  \begin{dfntn}
    Xét $a,b \in \mathbb{Z}$. Ta nói 
    \begin{quotation}
      $b$ là ước của $a$, hay \\
      $a$ chia hết cho $b$
    \end{quotation}
nếu có một số nguyên $c$ sao cho $$a = bc.$$
Ta viết  $b\mid a $ để chỉ $a$ chia hết cho $b$. Nếu $a$ không chia hết cho $b$ thì ta viết $b \nmid a$.
  \end{dfntn}
\pause 
  \begin{xmpl}
    \begin{itemize}
    \item $847 \mid 485331$ vì $485331 = 847 \cdot 573$.
    \item $355 \nmid 259943$ vì $259943 = $ 
    \end{itemize}
  \end{xmpl}
\end{frame}

\begin{frame}
  \begin{prpstn}
    Xét $a,b,c \in \Z$.
    \begin{enumerate}
    \item Nếu $a\mid b$ và $b\mid c$, thì $a\mid c$.
    \item Nếu $a\mid b$ và $b\mid a$, thì $a=\pm b$.
    \item Nếu $a \mid b$ và $a \mid c$, thì $a\mid (b + c)$ và $a\mid (b-c)$.
    \end{enumerate}
  \end{prpstn}
\end{frame}

\begin{frame}
  \begin{xrcs}
    Hãy chứng minh mệnh đề trước.
  \end{xrcs}
\end{frame}

\begin{frame}
  \begin{dfntn}
    \begin{itemize}
    \item Ước chung của hai số nguyên $a$ và $b$ là số nguyên $d$ thỏa
      mãn: $$d \mid a\quad \text{và}\quad  d\mid b.$$
    \item Ta ký hiệu $\gcd(a,b)$ là ước chung \alert{lớn nhất} của $a$ và $b$.
    \end{itemize}
  \end{dfntn}
\pause 
  \begin{xmpl}
    \begin{itemize}
    \item $\gcd(12,18) = 6$ vì $6\mid 12$ và $6\mid 18$ và không có số nào lớn hơn có tính chất này.
    \item $\gcd(748,2014) = 44$ vì 
      \begin{align*}
        \text{các ước của } 748 &= \{1, 2, 4, 11, 17, 22, 34, 44, 68, 187, 374, 748\},\\
        \text{các ước của } 2024 &=\{1, 2, 4, 8, 11, 22, 23, 44, 46, 88, 92, 184, 253,\\
 &\qquad\qquad \qquad \qquad \qquad \qquad\quad      506, 1012, 2024\}.
      \end{align*}
    \end{itemize}

\vspace{-0.5cm}

  \end{xmpl}
\end{frame}

\begin{frame}{Một số tính chất của hàm $\gcd$}
\begin{align*}
  \gcd(a,b) &= \gcd(b,a)\\
  \gcd(a,b) &= \gcd(-a,b)\\
  \gcd(a,0) &=  |a|\\
  \gcd(a,ka) &= |a| \qquad \text{ với mọi } k \in \Z.\\
\end{align*}
  
\end{frame}
\begin{frame}
  \begin{dfntn}[Chia lấy dư]
    Xét $a,b$ là các số nguyên dương. Ta nói $a$ chia cho $b$ có thương là $q$ và phần dư là $r$ nếu 
$$
a = b\cdot q + r \qquad \text{với }\qquad 0\leq r < b.
$$
  \end{dfntn}

  \begin{xrcs}
    Hãy chứng minh rằng các số $q$ và $r$ ở trên xác định  duy nhất bởi $a$ và $b$.
  \end{xrcs}
\end{frame}

\begin{frame}{Thuật toán  tính $\gcd(a,b)$}
  \begin{itemize}
  \item Chia $a$ cho $b$ ta được
    $$
    a = b\cdot q + r \qquad \text{với}\qquad 0\leq r < b.
    $$
  \item Áp dụng đẳng thức  
    $$
    \gcd(a,b) = \gcd(b,r).
    $$
  
  \end{itemize}
\end{frame}
\begin{frame}{Ví dụ: Tính $\gcd(2024,748)$}
  \begin{align*}
    2024 &= 748 \cdot 2 + 528\\
    748 &= 528 \cdot  1 + 220\\
    528 &= 220 \cdot  2 + 88\\
    220 &= 88 \cdot  2 + 44 \qquad  \leftarrow \qquad \gcd = 44\\
    88 &= 44 \cdot  2 + 0
  \end{align*}
\end{frame}


\begin{frame}
  \begin{thrm}[Thuật toán Euclid] Xét $a,b$ là hai số nguyên dương với $a\geq b$. Thuật toán sau đây tính $\gcd(a,b)$ sau một số hữu hạn bước.
    \begin{enumerate}
    \item Đặt $r_0 = a$ và $r_1 = b$.
    \item Đặt $i = 1$.
    \item\label{Ec3} Chia $r_{i-1}$ cho $r_i$, ta được 
      $$
      r_{i-1} = r_i \cdot q_i + r_{i+1} \qquad \text{với} \qquad 0\leq r_{i+1} < r_i. 
      $$
    \item Nếu $r_{i+1} = 0$, vậy thì 
      $$
      r_i = \gcd(a,b)
      $$
      và thuật toán kết thúc.
    \item Ngược lại, $r_{i+1} > 0$, vậy thì đặt $i=i+1$ và quay lại Bước~\ref{Ec3}.  
    \end{enumerate}
    
  \end{thrm}
\end{frame}

\begin{frame}
  \begin{thrm}
     Phép chia (Bước~\ref{Ec3}) của  Thuật toán Euclid thực hiện nhiều nhất 
    $$
    \log_2(b) + 2\quad  \text{ lần}.
    $$  
  \end{thrm}
\end{frame}

\begin{frame}
  \begin{block}{Thuật toán Euclid (dạng đệ quy)}
    \qquad $\text{EUCLID} (a,b)$ \\
    \qquad \qquad \textbf{if} $b == 0$ \\
    \qquad \qquad \qquad \textbf{return } $a$\\
    \qquad \qquad \textbf{else} \\
    \qquad \qquad \qquad \textbf{return}  $\text{EUCLID} (b,a \mod b)$ 
  \end{block}
\end{frame}

\begin{frame}{Thuật toán Euclid mở rộng}
  \begin{itemize}
  \item Thuật toán Euclid có thể mở rộng để tìm thêm một số thông tin.
  \item Cụ thể, chúng ta mở rộng thuật toán để tính thêm hệ số $x,y$ thỏa mãn 
    $$
    d = \gcd (a,b) = ax + by. 
    $$
  \item Các hệ số $x,y$ có thể âm hoặc bằng $0$. Các hệ số này sẽ có ích sau này khi tích phần tử nghịch đảo trong số học modun.
  \end{itemize}
\end{frame}

% \begin{frame}{Ví dụ}
% Ta tính toán $\gcd(345,92)$ như sau:
% \begin{align*}
%   345 = 92\cdot 3 + 69           &\longleftrightarrow\  &            a = &bq_1 + r_1 \\
%   92  = 69 \cdot 1 + \alert{23}  &\longleftrightarrow\  &            b = &r_1q_2 + \alert{r_2} \\
%   69 = 23\cdot 3 + 0             &\longleftrightarrow\  &          r_1 = &r_2q_3 + \alert{r_3} 
% \end{align*}

% Ta được dạng sau đây từ hai phương trình đầu tiên 
% \begin{align*}
%   23 &= 92 - 69\cdot 1 &\longleftrightarrow  r_2 &= b -  r_1q_2 \\
%      &= 92 - (345 - 92\cdot 3)\cdot 1  &\longleftrightarrow  &= b - (a - bq_1)q_2 \\
%     &= 4\cdot 92 + (-1)\cdot 345 &\longleftrightarrow  &= (1+q_1q_2)b + (-q_2)a 
% \end{align*}

%\end{frame}

\begin{frame}{Thuật toán Euclid mở rộng}
  \begin{itemize}
  \item \textit{Input} : Cặp số nguyên dương $(a,b)$
  \item \textit{Output}: Bộ ba $(d,x,y)$ thỏa mãn 
    $$
    d = \gcd (a,b) = ax + by. 
    $$ 
  \end{itemize}
    \begin{block}{}
     $\text{EXTENDED-EUCLID} (a,b)$ \\
     \qquad \textbf{if} $b == 0$ \\
     \qquad \qquad \textbf{return } $(a,1,0)$\\
     \qquad \textbf{else  } \\
     \qquad \qquad $(d',x',y') = \text{EXTENDED-EUCLID} (b,a \mod b)$\\     
     \qquad \qquad $(d,x,y) = (d',y',x' - \lfloor a/b\rfloor y' )$\\
     \qquad \qquad \textbf{ return } $(d,x,y)$     
  \end{block}
\end{frame}

\begin{frame}{Tính đúng đắn của thuật toán}
  \begin{itemize}
  \item Thuật toán tìm $(d,x,y)$ thỏa mãn 
    $$
    d = \gcd(a,b) = ax + by
    $$
  \item Nếu $b = 0$, vậy thì 
    $$
    d = a = a\cdot 1 + b\cdot 0.
    $$
  \item Nếu $b \not= 0$, thuật toán EXTENDED-EUCLID sẽ tính $(d', x',y')$ thỏa mãn 
    \begin{align*}
      d'=d &= \gcd (b, a \mod b)\\
           &= b x' + (a \mod b)y'
    \end{align*}
  \item Và vậy thì 
    \begin{align*}
      d &=  b'x' + (a - b \lfloor a/b\rfloor) y' \\
        &= a y' + b (x' - \lfloor a/b \rfloor y')
    \end{align*}
  \end{itemize}
\end{frame}

\begin{frame}{Ví dụ}
  \begin{block}{}
      \begin{table}
        \centering
      \begin{tabular}{cccccc}
        $a$ &$b$ &$\lfloor a/b \rfloor$& $d$& $x$& $y$\\
        \hline 
        $99$ &$78$ &$1$ &$3$ &$-11$ &$14$ \\
        $78$ &$21$ &$3$ &$3$ &$3$ &$-11$ \\
         $21$ &$15$ &$1$ &$3$ &$-2$ &$3$ \\
        $15$ &$6$ &$2$ &$3$ &$1$ &$-2$ \\
        $6$ &$3$ &$2$ &$3$ &$0$ &$1$ \\
        $3$ &$0$ &$-$ &$3$ &$1$ &$0$ \\
        \hline 
      \end{tabular}
    \end{table}
\vspace{-0.5cm}
 
  \end{block}
  \begin{itemize}
  \item Mỗi dòng  của bảng mô tả một mức đệ quy: các giá trị đầu vào $a$ và $b$, giá trị tính $\lfloor a/b \rfloor $, và giá trị trả về  $d , x, y$. 
  \item Bộ ba $d, x, y$ được trả về trở thành bộ ba $d', x',y'$ của mức tiếp theo. 
  \item Lời gọi  thủ tục EXTENDED-EUCLID(99,78) trả về $(3,-11,4)$ thỏa mãn $\gcd(99,78) = 3 = 99 \cdot (-11) + 78\cdot 14$.
  \end{itemize}
% Vậy ta có 
% $$
% 3 = \gcd (99,78) = 99 \cdot (-11) + 79\cdot 14
% $$
\end{frame}

\begin{frame}
  \begin{xrcs}
    Hãy tính giá trị $$(d,x,y) = \text{EXTENDED-EUCLID}(899,493).$$ 
  \end{xrcs}
\end{frame}

\section{Số học đồng dư} 
\begin{frame}
  \begin{dfntn}
    Xét số nguyên $m \geq  1$. Ta nói hai số nguyên $a$ và $b$ là đồng dư theo modun $m$ nếu $a-b$ chia hết cho $m$, và viết
    $$
    a \equiv b \pmod{m}
    $$
Số $m$ được gọi là modun.
  \end{dfntn}
Đồng hồ có thể được viết theo như modun dùng modun  $m = 12$:
$$
6+9 = 15 \equiv 3 \pmod{12}\qquad \text{và}\qquad 2 - 3 = - 1 \equiv 11  \pmod{12}  
$$
\end{frame}

\begin{frame}
  \begin{xmpl}
    \begin{itemize}
    \item $17 \equiv 7 \pmod{5}$ vì $10 = 17 - 7$ chia hết cho $5$.
    \item $19 \not\equiv 6 \pmod{11} $ vì  $19 - 6 = 13$ không chia hết cho $11$
    \end{itemize}
  \end{xmpl}
\end{frame}

\begin{frame}
  \begin{prpstn}
    Xét số nguyên $m\geq 1$. 
    \begin{enumerate}
    \item<+-> Nếu $a_1 \equiv a_2 \pmod{m}$ và $b_1 \equiv b_2 \pmod{m}$, vậy thì 
      \begin{align*}
      a_1\pm b_1 &\equiv a_2\pm b_2 \pmod{m},\quad  \text{và}\\
        a_1\cdot b_1 &\equiv a_2\cdot b_2 \pmod{m}.
      \end{align*}
    \item<+-> Xét số nguyên $a$. Vậy thì tồn tại số nguyên $b$ thỏa mãn 
      $$
      a\cdot b \equiv 1 \pmod{m} \text{ \alert{nếu và chỉ nếu}  } \gcd(a,m) = 1.
      $$
      Nếu tồn tại số $b$ như vậy thì ta nói $b$ là nghịch đảo của $a$ theo modun $m$.
    \end{enumerate}
  \end{prpstn}
\end{frame}

\begin{frame}
  \begin{xrcs}
    \begin{itemize}
    \item<+-> Lấy $m = 5$ và $a=2$. Rõ ràng $\gcd(2,5) = 1$, vậy thì tồn
      tại nghịch đảo của $a$ theo modun $5$. Hãy tìm $a^{-1}$.
    \item<+-> Tương tự $\gcd(4,15) = 1$. Hãy tìm $4^{-1}$ theo modun $15$.
    \end{itemize}
  \end{xrcs}
\end{frame}

\begin{frame}
  \begin{dfntn}
    Ta viết 
    $$
    \Z/m\Z = \{0,1,2,\dots, m-1\}
    $$
và gọi $\Z/m\Z$ là vành số nguyên modun $m$. 
  \end{dfntn}
  \begin{rmrk}
    Khi chúng ta thực hiện phép cộng hoặc nhân trong $\Z/m\Z$ ta luôn chia kết quả cho $m$ và lấy phần dư.
  \end{rmrk}
\end{frame}

\begin{frame}
  \begin{table}
    \centering
    $\begin{tabu}{|c||c|c|c|c|c|}
\hline 
      +& 0& 1& 2& 3& 4\\
      \hline \hline
      0 &0 &1 &2 &3 &4\\
      \hline 
      1& 1& 2& 3& 4& 0 \\
      \hline 
      2 & 2 & 3 &4 &0 &1 \\
      \hline 
      3 & 3& 4 & 0 & 1&2 \\
      \hline 
      4 & 4& 0 &1 &2 &3\\
      \hline 
    \end{tabu}$ \qquad \qquad 
    $\begin{tabu}{|c||c|c|c|c|c|}
\hline 
      \cdot & 0& 1& 2& 3& 4\\
      \hline \hline
      0 &0 &0 &0 &0 &0\\
      \hline 
      1& 0& 1& 2& 3& 4 \\
      \hline 
      2 & 0 & 2 &4 &1 &3 \\
      \hline 
      3 & 0& 3 & 1 & 4&2 \\
      \hline 
      4 & 0& 4 &3 &2 &1\\
      \hline 
    \end{tabu}$

    \caption{Cộng và nhân theo modun 5}
  \end{table}
\end{frame}

\begin{frame}
  \begin{dfntn}
    Ta biết rằng $a$ có nghịch đảo modun $m$
 nếu và chỉ nếu $\gcd(a,m) = 1$. Các số khả nghịch gọi là \textit{đơn vị}. Ta ký hiệu tập mọi đơn vị bởi 
 \begin{align*}
   (\Z/m\Z)^* &= \{a \in \Z/m\Z : \gcd(a,m) = 1 \} \\
              &=  \{a \in \Z/m\Z : a \text{ có nghịch đảo theo modun  } m \}
 \end{align*}
Tập $(\Z/m\Z)^*$ được gọi là \alert{nhóm đơn vị theo modun $m$}. 
  \end{dfntn}
\end{frame}

\begin{frame}
  \begin{xmpl}
    Nhóm đơn vị theo modun $24$ là
    $$
    (\Z/24\Z)^* = \{1,5,7,11,13,17,19,23\}.
    $$
Bảng nhân của nhóm này xác định như sau:
%\begin{center}
  $$\begin{tabu}{|c||c|c|c|c|c|c|c|c|}
\hline 
\cdot &1 &5 &7 &11 &13 &17 &19 &23 \\
\hline \hline 
1 &1 &5 &7 &11 &13 &17 &19 &23\\
\hline 
5 &5 &1 &11 &7 &17 &13 &23 &19\\
\hline 
7 &7 &11 &1 &5 &19 &23 &13 &17\\
\hline 
11 &11 &7 &5 &1 &23 &19 &17 &13\\
\hline 
13 &13 &17 &19 &23 &1 &5 &7 &11\\
\hline 
17 &17 &13 &23 &19 &5 &1 &11 &7\\
\hline 
19 &19 &23 &13 &17 &7 &11 &1 &5\\
\hline 
23 &23 &19 &17 &13 &11 &7 &5 &1\\
\hline 
  \end{tabu}$$
%\end{center}
  \end{xmpl}
\end{frame}
\begin{frame}
  \begin{xmpl}
    Nhóm đơn vị theo modun $7$ là
    $$
    (\Z/7\Z)^* = \{1,2,3,4,5,6\}
    $$
vì các  từ $1$ đến $6$ đều nguyên tố cùng nhau với $7$. Bảng nhân của nhóm này được xác định như dưới đây.
  $$\begin{tabu}{|c||c|c|c|c|c|c|}
\hline 
\cdot &1 &2 &3 &4 &5 &6\\
\hline 
1 &1 &2 &3 &4 &5 &6\\
\hline 
2 &2 &4 &6 &1 &3 &5\\
\hline 
3 &3 &6 &2 &5 &1 &4\\
\hline 
4 &4 &1 &5 &2 &6 &3\\
\hline 
5 & 5 &3 &1 &6 &4 &2\\
\hline 
6 &6 &5 &4 &3 &2 &1\\
\hline 
  \end{tabu}
  $$
  \end{xmpl}
\end{frame}

\begin{frame}
  \begin{dfntn}
    Phi hàm Euler là hàm $\phi(m)$ định nghĩa bởi luật 
    \begin{align*}
    \phi(m) &= \#(\Z/m\Z)^*\\ 
            &= \#\{0\leq a <m : \gcd (a,m)=1\}.
    \end{align*}
  \end{dfntn}
Ví dụ $$\phi(24) = 8\qquad  \text{và}\qquad \phi(7) = 6.$$
\end{frame}

\begin{frame}{Tính lũy thừa nhanh}
  \begin{xmpl}
    Giả sử ta muốn tính 
    $$
    3^{218} \pmod{1000}.
    $$
    Đầu tiên, ta viết $218$ ở dạng cơ số $2$:
    $$
    218 = 2 + 2^3 + 2^4 + 2^6 + 2^7.
    $$
    Vậy thì $3^{218}$ trở thành 
    $$
    3^{218} = 3^{2 + 2^3 + 2^4 + 2^6 + 2^7}=3^2\cdot 3^{2^3} \cdot  3^{2^4} \cdot 3^{2^6} \cdot 3^{2^7}. 
    $$
Để ý rằng, dễ tính các mũ 
$$
3\ ,\ 3^2\ ,\ 3^{2^2}\ , \  3^{2^3}\ , \ 3^{2^4}\ ,\ \dots  
$$
  \end{xmpl}
\end{frame}

\begin{frame}
  \begin{xmpl}[tiếp]
   Ta lập bảng 
   $$\begin{tabu}{|c||c|c|c|c|c|c|c|c|} 
     \hline 
     i & 0 & 1& 2 &3 &4 &5 &6 &7\\
     \hline 
     3^{2^i}\pmod{1000}  &3  &9 &81 &561 &721 &841 &281 &961\\
\hline 
   \end{tabu}$$
rồi tính 
   \begin{align*}
     3^{218} &= 3^2\cdot 3^{2^3} \cdot  3^{2^4} \cdot 3^{2^6} \cdot 3^{2^7}\\
            &\equiv  9 \cdot 561 \cdot 721 \cdot 281 \cdot 961 \pmod{1000}\\
            &\equiv 489 \pmod{1000}.
   \end{align*}
  \end{xmpl}
\end{frame}

\begin{frame}{Thuật toán tính nhanh $a^b \pmod{n}$}% {Thuật toán bình phương liên tiếp}
  % \begin{itemize}
  % \item \textit{Input}: Bộ ba $(a,b,n)$
  % \item \textit{Output}: Giá trị $a^b \pmod{n}$
  % \end{itemize}
  \begin{block}{}
    MODULAR-EXPONENTIATION($a,b,n$)\\
    \qquad $c= 0$\\
    \qquad $d = 1$\\
    \qquad Biểu diễn  $ b = \langle b_k, b_{k-1}, \dots, b_0\rangle_2$ \\
    \qquad \textbf{for } $i = k$ \textbf{ downto } $0$\\
    \qquad \qquad $c = 2c$\\
    \qquad \qquad $d = (d\cdot d)\pmod{n}$\\
    \qquad \qquad \textbf{if } $b_i == 1$\\
    \qquad \qquad \qquad $c= c+ 1$\\
    \qquad \qquad \qquad $d=(d \cdot a) \pmod{n}$\\
    \qquad \textbf{return } $d$
  \end{block}
\end{frame}

\begin{frame}{Ví dụ}
  \[    
  \begin{tabu}{c|cccccccccc}
    i&9&8&7&6&5&4&3&2&1&0 \\
\hline 
    b_i &1&0&0&0&1&1&0&0&0&0 \\
    c &1&2&4&8&17&35&70&140&280&560 \\
    d &7&49&157&526&160&241&298&166&67&1 \\
  \end{tabu}
  \]
  \begin{itemize}
  \item Kết quả tính $a^b\pmod{n}$ với 
    $$
    a=7, \quad b=560=\langle 1000110000 \rangle, \text{ và } n = 561
    $$
  \item Giá trị được chỉ ra sau mỗi bước lặp. 
  \item Kết quả cuối cùng bằng $1$
  \end{itemize}
\end{frame}

\section{Số nguyên tố và trường hữu hạn}
%\begin{frame}
\begin{frame}
  \begin{dfntn}
    \begin{itemize}
    \item<+-> \alert{Số nguyên tố} là số nguyên lớn hơn $1$, không chia hết
      cho số nguyên dương nào ngoài $1$ và chính nó.
    \item<+-> Số nguyên lớn hơn
      $1$ không phải số nguyên tố được gọi là \alert{hợp~số}.
    \end{itemize}
  \end{dfntn}
\end{frame}

\begin{frame}{$100$ số nguyên tố đầu tiên}
  \begin{center}
    \begin{tabular}[h]{cccccccccc}
      2& 3& 5& 7& 11& 13& 17& 19& 23& 29\\
      31& 37& 41& 43& 47& 53& 59& 61& 67 &71\\
      73& 79& 83& 89& 97& 101& 103& 107& 109 &113\\
      127& 131& 137& 139& 149& 151& 157& 163& 167 &173\\
      179& 181& 191& 193& 197& 199& 211& 223& 227 &229\\
      233& 239& 241& 251& 257& 263& 269& 271& 277 &281\\
      283& 293& 307& 311& 313& 317& 331& 337& 347 &349\\
      353& 359& 367& 373& 379& 383& 389& 397& 401 &409\\
      419& 421& 431& 433& 439& 443& 449& 457& 461 &463\\
      467& 479& 487& 491& 499& 503& 509& 521& 523 &541
      % 547,
                                                    

      % 557, 563, 569, 571, 577, 587, 593, 599, 601, 607, 613, 617,
      % 619, 631, 641, 643, 647, 653, 659, 661, 673, 677, 683, 691,
      % 701, 709, 719, 727, 733, 739, 743, 751, 757, 761, 769, 773,
      % 787, 797, 809, 811, 821, 823, 827, 829, 839, 853, 857, 859,
      % 863, 877, 881, 883, 887, 907, 911, 919, 929, 937, 941, 947,
      % 953, 967, 971, 977, 983, 991, 997, 1009, 1013, 1019, 1021,
      % 1031, 1033, 1039, 1049, 1051, 1061, 1063, 1069, 1087, 1091,
      % 1093, 1097, 1103, 1109, 1117, 1123, 1129, 1151, 1153, 1163,
      % 1171, 1181, 1187, 1193, 1201, 1213, 1217, 1223, 1229, 1231,
      % 1237, 1249, 1259, 1277, 1279, 1283, 1289, 1291, 1297, 1301,
      % 1303, 1307, 1319, 1321, 1327, 1361, 1367, 1373, 1381, 1399,
      % 1409, 1423, 1427, 1429, 1433, 1439, 1447, 1451, 1453, 1459,
      % 1471, 1481, 1483, 1487, 1489, 1493, 1499, 1511, 1523, 1531,
      % 1543, 1549, 1553, 1559, 1567, 1571, 1579, 1583, 1597, 1601,
      % 1607, 1609, 1613, 1619, 1621, 1627, 1637, 1657, 1663, 1667,
      % 1669, 1693, 1697, 1699, 1709, 1721, 1723, 1733, 1741, 1747,
      % 1753, 1759, 1777, 1783, 1787, 1789, 1801, 1811, 1823, 1831,
      % 1847, 1861, 1867, 1871, 1873, 1877, 1879, 1889, 1901, 1907,
      % 1913, 1931, 1933, 1949, 1951, 1973, 1979, 1987, 1993, 1997,
      % 1999, 2003, 2011, 2017, 2027, 2029, 2039, 2053, 2063, 2069,
      % 2081, 2083, 2087, 2089, 2099, 2111, 2113, 2129, 2131, 2137,
      % 2141, 2143, 2153, 2161, 2179, 2203, 2207, 2213, 2221, 2237,
      % 2239, 2243, 2251, 2267, 2269, 2273, 2281, 2287, 2293, 2297,
      % 2309, 2311, 2333, 2339, 2341, 2347, 2351, 2357, 2371, 2377,
      % 2381, 2383, 2389, 2393, 2399, 2411, 2417, 2423, 2437, 2441,
      % 2447, 2459, 2467, 2473, 2477, 2503, 2521, 2531, 2539, 2543,
      % 2549, 2551, 2557, 2579, 2591, 2593, 2609, 2617, 2621, 2633,
      % 2647, 2657, 2659, 2663, 2671, 2677, 2683, 2687, 2689, 2693,
      % 2699, 2707, 2711, 2713, 2719, 2729, 2731, 2741
    \end{tabular}
  \end{center}
\end{frame}


\begin{frame}
  \begin{prpstn}
    Xét số nguyên tố  $p$, và giả sử rằng tích $ab$ của hai số $a$ và $b$ chia hết cho~$p$. Vậy thì $a$ hoặc $b$ phải chia hết cho $p$.\\

Tổng quát hơn nếu 
$$
p \mid a_1 a_2\cdots a_n, 
$$ 
vậy thì ít nhất một trong các số $a_i$ phải chia hết cho $p$.
  \end{prpstn} 
\end{frame}

\begin{frame}
  \begin{xrcs}
    Hãy chứng minh mệnh đề trước.
  \end{xrcs}
\end{frame}

\begin{frame}
  \begin{thrm}[Định lý cơ bản của số học]
    Mọi số nguyên $a\geq 2$ đều phân tích được thành tích các số nguyên tố
    $$
    a = p_1^{e_1}\cdot p_2^{e_2}\cdot p_3^{e_3}\cdots p_r^{e_r}.
    $$
    Hơn nữa phân tích này là duy nhất nếu  các thừa số được viết với thứ tự không giảm. 
  \end{thrm}
\end{frame}

\begin{frame}
  \begin{xrcs}
    Hãy chứng minh định lý trước.
  \end{xrcs}
\end{frame}
\begin{frame}
  \begin{dfntn}
    \begin{itemize}
    \item Định lý cơ bản của số học chỉ ra rằng trong phân tích thừa
      số nguyên tố của số nguyên dương $a$, mỗi số nguyên tố $p$ xuất
      hiện với một số mũ nào đó. 
    \item Ta ký hiệu số mũ này là
      \alert{$\ord_p(a)$} và gọi nó là \alert{cấp} (hoặc \alert{số
        mũ}) của $p$ trong $a$.
    \item Để cho tiện, ta kí hiệu $\ord_p(1) = 0$ với mọi số nguyên tố $p$.
    \end{itemize}
  \end{dfntn}
\end{frame}

\begin{frame}
  \begin{xmpl}
    Phân tích của $1728$ là 
    $$
    1728 = 2^6 \cdot 3^3.
    $$
Vậy thì 
$$
\ord_2(1726) = 6,\qquad  \ord_3(1726) = 3,
$$
và 
$$
 \ord_p(1728) = 0 \text{ với mọi số nguyên tố } p\geq 5.  
$$
  \end{xmpl}
\end{frame}

\begin{frame}
  \begin{prpstn}
    Xét số nguyên tố $p$. Khi đó  mọi phần tử $a$ khác $0$ của $\Z/p\Z$ đều có nghịch đảo, có nghĩa rằng, tồn tại $b$ để 
    $$  
    ab \equiv  1 \pmod{p}.
    $$
Ta ký hiệu giá trị $b$ này bởi $a^{-1}\bmod  p$, hoặc đơn giản là $a^{-1}$ nếu $p$ đã xác định.
  \end{prpstn}
Mệnh đề này chỉ ra rằng 
$$
(\Z/p\Z)^* = \{1,2,3,4,\dots, p-1\}.
$$
\end{frame}

\begin{frame}
  \begin{xrcs}
    Hãy chỉ ra thuật toán tính phần tử nghịch đảo $a^{-1}$ của phần tử $a$ trong nhóm $(\Z/p\Z)^*$.
  \end{xrcs}
\end{frame}

\begin{frame}{Trường hữu hạn $\F_p$}
  \begin{itemize}
  \item Nếu $p$ nguyên tố, vậy thì tập $\Z/p\Z$ với phép toán cộng, trừ, nhân và luật chia là  một \alert{trường}.
  \item Trường $\Z/p\Z$ chỉ có hữu hạn phần tử. Đây là trường hữu hạn và ta ký hiệu $\F_p$.
  \item Ta viết $(\F_p)^*$ cho nhóm $(\Z/p\Z)^*$.
  \item Trong $\F_p$ người ta thường ký hiệu 
    $$
    a = b \text{ thay cho } a\equiv b \pmod{p}. 
    $$ 
  \end{itemize}
\end{frame}
\section{Lũy thừa và căn nguyên thủy trong trường hữu hạn}
\begin{frame}{Ví dụ}
%\setlength{\tabcolsep}{12pt}
  \begin{table}
    \centering%
  $$\begin{tabu}{cccccc}
    1^1\equiv 1&1^2\equiv 1&1^3\equiv 1&1^4\equiv 1&1^5\equiv 1&1^6\equiv 1\\
2^1\equiv 2&2^2\equiv 4&2^3\equiv 1&2^4\equiv 2&2^5\equiv 4&2^6\equiv 1\\
3^1\equiv 3&3^2\equiv 2&3^3\equiv 6&3^4\equiv 4&3^5\equiv 5&3^6\equiv 1\\
4^1\equiv 4&4^2\equiv 2&4^3\equiv 1&4^4\equiv 4&4^5\equiv 2&4^6\equiv 1\\
5^1\equiv 5&5^2\equiv 4&5^3\equiv 6&5^4\equiv 2&5^5\equiv 3&5^6\equiv 1\\
6^1\equiv 6&6^2\equiv 1&6^3\equiv 6&6^4\equiv 1&6^5\equiv 6&6^6\equiv 1
  \end{tabu}$$
  \caption{Các lũy thừa theo theo modun $7$}
\end{table}
  \begin{qstn}
    Tại sao cột bên tay phải toàn nhận giá trị $1$?
  \end{qstn}
\end{frame}

\begin{frame}
  \begin{thrm}[Định lý Fermat nhỏ]
    Xét số nguyên tố $p$ và xét số nguyên $a$. Khi đó 
    $$
    a^{p-1} \equiv
    \begin{cases}
      1 \pmod{p} &\text{nếu } p \nmid a, \\
      0 \pmod{p} &\text{nếu } p \mid a.
    \end{cases}
    $$
  \end{thrm}
\end{frame}

\begin{frame}
  \begin{xmpl}
    Số $p = 15485863$ là số nguyên tố, vậy thì 
    $$
    2^{15485862} \equiv 1 \pmod{15485863}.
    $$
Vậy thì, không cần tính toán ta cũng biết rằng 
\begin{quote}
  số $2^{15485862} - 1$ là bội số của $15485863$.
\end{quote}
  \end{xmpl}
\end{frame}

\begin{frame}
  \begin{rmrk}
    Định lý Fermat nhỏ và thuật toán tính nhanh lũy thừa cho ta một phương pháp hợp lý để tính nghịch đảo theo modun $p$. Cụ thể
    $$
    a^{-1}\equiv a^{p-2} \pmod {p}.
    $$
    Thời gian tính toán của phương pháp này tương tự như dùng thuật toán Euclid mở rộng.
  \end{rmrk}
\end{frame}
\begin{frame}
  \begin{dfntn}
    \alert{Cấp của phần tử $a$ theo modun $p$} là số mũ $k > 0$ nhỏ nhất thỏa mãn 
    $$
    a^k \equiv 1 \pmod{p}.
    $$ 
  \end{dfntn}
  \begin{prpstn}
    Xét số nguyên tố $p$ và xét số nguyên $a$ không chia hết cho $p$. Giả sử $a^n\equiv 1 \pmod{p}$. Vậy thì $n$ chia hết cho cấp của $a$ theo modun $p$. Đặc biệt, $p-1$ chia hết cho cấp của $a$.
  \end{prpstn}
\end{frame}

\begin{frame}
  \begin{xrcs}
    Hãy chứng minh mệnh đề trước.
  \end{xrcs}
\end{frame}

\begin{frame}
  \begin{thrm}[Định lý căn nguyên thủy]
    Xét số nguyên tố $p$. Khi đó có tồn tại một phần tử $g \in \F_p^*$ thỏa mãn mọi phần tử của $\F_p^*$ đều là một lũy thừa nào đó của $g$. Tức là 
    $$
    \F_p^* = \{1, g, g^2, g^3, \cdots, g^{p-2}\}.
    $$ 
Các phần tử có tính chất này được gọi là căn nguyên thủy của $\F_p$ hoặc phần tử sinh của $\F_p^*$. Chúng là các phần tử của $\F_p^*$ có cấp $p-1$.
  \end{thrm}
\end{frame}

\begin{frame}
  \begin{xmpl}
    Trường $\F_{11}$ có $2$ là một căn nguyên thủy, bởi vì trong $\F_{11}$,
    $$\begin{tabu}{ccccc}
      2^0 = 1 &  2^1=2 & 2^2 = 4 & 2^3 = 8  &2^4 = 5\\
      2^5 =10 &  2^6=9 & 2^7 = 7 & 2^8 = 3  & 2^9 = 6. 
    \end{tabu}$$
    nhưng $2$ không phải căn nguyên thủy của $\F_{17}$, bởi vì trong $\F_{17}$
    $$\begin{tabu}{ccccc}
      2^0 = 1 & 2^1= 2&2^2= 4&2^3= 8&2^4 =  16\\
      2^5= 15&2^6= 13&2^7 = 9& 2^8 = 1
    \end{tabu}$$
  \end{xmpl}
\end{frame}

\begin{frame}
  \begin{xrcs}
    \begin{itemize}
    \item Hãy tìm một căn nguyên thủy của trường $\F_{17}$.
    \item Hãy liệt kê tất cả các căn nguyên thủy của $\F_{17}$.
    \end{itemize}

  \end{xrcs}
\end{frame}
\end{document}



%%% Local Variables:
%%% mode: latex
%%% TeX-master: t
%%% End:
