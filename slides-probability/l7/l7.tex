\documentclass[mathserif
, handout
]{beamer}
 
 % \useoutertheme{wuerzburg}
%  \useinnertheme[outline]{chamfered}
\usepackage{rotating}
\usepackage[]{algorithm2e}
\usepackage{color, colortbl}
%\usepackage{default}
\usepackage{fontspec}
\usepackage{polyglossia} 
\setmainlanguage{vietnamese}
%\setdefaultlanguage{vietnamese} 
%\setmainfont{Palatino}
 \usepackage{wasysym}
\usepackage{pifont}% http://ctan.org/pkg/pifont

\usepackage{multicol}
\usepackage{sidecap}

\usepackage{hyperref}

\usepackage{pgf}    
\usepackage{tikz}
\usetikzlibrary{arrows,automata,decorations.pathmorphing,backgrounds,positioning,fit}
\usepackage{array}
\usepackage{listings}

\usepackage{enumerate}
%\usepackage{amsmath,mathtools}
%		\usepackage{fink}

\usepackage{amsmath,amsthm, amssymb}
\usepackage{microtype}
\usetikzlibrary{arrows,automata}
\usetikzlibrary{decorations.pathmorphing}


\usetikzlibrary{calc}

 

\usetikzlibrary{trees}
\usepackage{listings}

\setbeamertemplate{footline}[frame number]
\setbeamertemplate{navigation symbols}{}%remove navigation symbols

%\usepackage{listingsutf8}

%\setbeameroption{show notes on second screen=right}


  
%\newtheorem{Lemme}{Bổ đề} 
%\newtheorem*{LI}{Lemme d'itération infinie}

%\newtheorem{Proposition}{Mệnh đề}[section]
%\newtheorem{Theorem}{Định lý}[section]
%\newtheorem{Corollaire}{Hệ quả}[section]
%\newtheorem*{Conjecture}{Giả thuyết}
% \newtheorem*{Probleme}{Bài toán}
% \newtheorem*{Fait}{Fait}
% 
% 
% \theoremstyle{definition} \newtheorem{Definition}{Định nghĩa}
% \theoremstyle{definition} \newtheorem{example}{Ví dụ}
% \theoremstyle{remark} \newtheorem*{Remarque}{Chú ý}


\usetikzlibrary{arrows,automata}



\usetikzlibrary{trees}


% \newcommand{\mnvect}[2]
% {
%   \begin{bmatrix}	#1\\#2
%   \end{bmatrix}
% }

\definecolor{olive}{rgb}{0.3, 0.4, .1}
\definecolor{fore}{RGB}{249,242,215}
\definecolor{back}{RGB}{51,51,51}
\definecolor{title}{RGB}{255,0,90}
\definecolor{dgreen}{rgb}{0.,0.7,0. }
\definecolor{gold}{rgb}{1.,0.84,0.}
\definecolor{JungleGreen}{cmyk}{0.99,0,0.52,0}
\definecolor{BlueGreen}{cmyk}{0.85,0,0.37,0}
\definecolor{RawSienna}{cmyk}{0,0.72,1,0.45}
\definecolor{Magenta}{cmyk}{0,1,0,0} 



%\include{./myDefs}

%\setlength{\topmargin}{0cm} \setlength{\oddsidemargin}{0cm}
%\setlength{\evensidemargin}{0cm} \setlength{\textwidth}{17truecm}
%\setlength{\textheight}{21.0truecm}


%\parindent = 3 pt
%\parskip = 12 pt

%\newtheorem*{LI}{Lemme d'itération infinie}



\newtheorem{prprt}{Propriété}
\newtheorem{prpstn}{Mệnh đề}
\newtheorem{thrm}{Định lý}
\newtheorem{lmm}{Bổ đề}
\newtheorem{rl}{Luật}

\newtheorem{crllr}{Hệ quả}
\newtheorem{clm}{Khẳng định}
\newtheorem{nt}{Notation}
 
\newtheorem*{cnjctr}{Giả thuyết}

\newtheorem{fct}{Fait}
%\newtheorem{xmpl}{Exemple}
\newtheorem{rmrk}{Nhận xét}

\theoremstyle{example}
\newtheorem{xmpl}{Ví dụ}
\newtheorem{xrcs}{Bài tập}
  \newtheorem{dfntn}{Định nghĩa}
  \newtheorem{qstn}{Câu hỏi}
\newtheorem{prblm}{Bài toán}  
   \newtheorem{sol}{Lời giải}
  
%  \newtheorem{rmrk}{Định nghĩa}
  

% \declaretheorem[name=Problème]{prblm}
% \declaretheorem[name=Question, style=remark, numbered=no]{qstn}

% \declaretheorem[name=Théorème, numberwithin=section]{thrm}
% \declaretheorem[name=Lemme, sibling=thrm]{lmm}
% \declaretheorem[ name=Propriété, sibling=thrm]{prprt}
% \declaretheorem[ name=Proposition, sibling=thrm]{prpstn}
% \declaretheorem[name=Corollaire, sibling=thrm]{crllr}
% \declaretheorem[name=Fait, sibling=thrm]{fct}
% \declaretheorem[name=Notation, sibling=thrm]{nt}


% \declaretheorem[style=definition, name=Définition, sibling=thrm]{dfntn}

% %\theoremstyle{definition} \newtheorem{dfntn}{Définition}[section]

% \renewcommand\thmcontinues[1]{reprise de p.\,\pageref{#1}}

% \declaretheorem[style=remark, name=Exemple%, numberwithin=section
% ]{xmpl}

% \declaretheorem[style=remark, name=Remarque, numbered=no]{rmrk}

% %\declaretheorem[style=definition,numberwithin=chapter,name = Exemple]{xmpl}

% %\theoremstyle{remark} \newtheorem{xmpl}{Exemple}[chapter]

% %\theoremstyle{remark} \newtheorem*{rmrk}{Remarque}



\newtheorem{cs}{Cas}


\def\mclose{\texttt{close}}
\def\mopen{\texttt{open}}

\def\mmclose{\texttt{\scriptsize close}}
\def\mmopen{\texttt{\scriptsize open}}



% \newcommand{\mvect}[2]
% {
% \bigl[ \begin{smallmatrix}
% #1\\ #2
% \end{smallmatrix} \bigr]
% }

% \newcommand{\mnvect}[2]
% {
%   \begin{bmatrix}	#1\\#2
%   \end{bmatrix}
% }

% % \newcommand{\mnvect}[2]
% % {
% % #1/#2
% %   % \begin{bmatrix}	#1\\#2
% %   % \end{bmatrix}
% % }

% \newcommand{\XMPL}[3]
% {
%   \begin{xmpl}
%     Soient $L=\{#1\}$ et $\Sigma=\{#2\}$. On peut vérifier que $L$ est \orl\ avec le
%     relateur de base $#3$.
%   \end{xmpl}
% }

% \newcommand{\XMP}[4]
% {
%   \begin{xmpl}[#4]
%     Soient $L=\{#1\}$ et $\Sigma=\{#2\}$. On peut vérifier que $L$ est \orl\ avec le
%     relateur de base $#3$.
%   \end{xmpl}
% }

% \newcommand{\Pui}[2]
% {
%   #1^{\leq #2}
% }


% % \newcommand{\XMPL1}[4]
% % {
% %   \begin{xmpl}
% %     Soient $L=\{#1\}$ et $\Sigma=\{#2\}$. Il est clair que $L$ est \orl\ avec le
% %     relateur de base $#3$. $L^\omega$ est un 
% %   \end{xmpl}
% % }

% \def\vvs{\vspace{11pt}}
% \def\nni{\noindent}


% \newcommand{\cas}[1]
% {
% \vvs\nni
% \textbf{Cas #1 :}
% }



% \newcommand{\souscas}[1]
% {
% \vvs\nni
% \textbf{Sous-cas #1 :}
% }

% \def\pcom{paire de mots incompatibles}
% \def\wpcom{paire de mots $\infty$-incompatible}

% \def\upcom{une paire de mots incompatibles}
% \def\uwpcom{une paire de mots $\infty$-incompatibles}
% \def\comp{\asymp}

% \def\wg{code générateur}

%  \def\gc{code générateur}

% \def\gcx{codes générateurs}
% \def\Gcx{Codes générateurs }
% \def\ugc{un code générateur}
% \def\Ugc{un Code générateur}

% \def\wgc{$\omega$-code générateur}
% \def\wgcx{$\omega$-codes générateurs}
% \def\wGcx{$\omega$-Codes générateurs }
% \def\wugc{un $\omega$-code générateur}
% \def\wUgc{un $\omega$-code générateur}

% \def\orl {langage à un relateur}

% \def\orlx {langages à un relateur}
% \def\Orlx {Langages à un relateur}
% \def\uorl {un langage à un relateur}


% \def\ugc{un code générateur}

% \def\cp{code préfixe}

% \def\iff{si et seulement si} 
% \def\w{\omega}

% \def\CODE{la proposition~\ref{c3prop23}, $L^\omega$ n'a pas de \gc}
% \def\NOCODE{$L^\omega$ n'a pas de \gc}


\def\vs{}
\def\ni{}


\def\trail{hành trình đơn}
\def\Trail{Hành trình đơn}

\def\ctrail{\trail\ đóng}
\def\Ctrail{\Trail\ đóng }

\def\walk{hành trình}
\def\Walk{Hành trình}

\def\cwalk{hành trình đóng}
\def\Cwalk{Hành trình đóng}

\def\path{đường đi}
\def\Path{Đường đi}
 
\def\conn{liên thông}
\def\Conn{Liên thông}

\def\Comp{Thành phần liên thông}
\def\comp{thành phần liên thông}

\def\Cuted{Cạnh cắt}
\def\cuted{cạnh cắt}

\def\Cutve{Đỉnh cắt}
\def\cutve{đỉnh cắt}

\def\Induced{Đồ thị con cảm sinh}
\def\induced{đồ thị con cảm sinh}

 
\def\iff{{\color{blue} nếu và chỉ nếu}}

\def\ideg{\text{indeg}}
\def\odeg{\text{outdeg}}

\def\pr{\mathrm{Pr}}
\def\ex{\mathrm{Ex}}
\def\S{\mathcal{S}}
\def\var{\mathrm{Var}}


 \newcommand{\defi}[1]{{\color{blue}{\textbf{\emph{#1}}}}}
\newcommand{\contradiction}{{\hbox{%
    \setbox0=\hbox{$\mkern-3mu\times\mkern-3mu$}%
    \setbox1=\hbox to0pt{\hss$\times$\hss}%
    \copy0\raisebox{0.5\wd0}{\copy1}\raisebox{-0.5\wd0}{\box1}\box0
}}}

\newcommand{\cmark}{{\color{blue}\Large\ding{51}}}%
\newcommand{\xmark}{{\color{red}\Large\ding{55}}}%

%\newcommand{\defi}[1]{{\color{blue}{\textbf{\emph{#1}}}}}


 \AtBeginSection[]  
 { 
   \begin{frame}[plain]{Nội dung} 
     \tableofcontents[currentsection,currentsubsection] 
   \end{frame} 
 }  



\begin{document}
% \tikzstyle{every picture}+=[remember picture]

% \tikzstyle{na} = [baseline=-.5ex]

\author{Trần Vĩnh Đức}
%\institute[HUST]{Hanoi University of Science and Technology}


% \newcommand{\cmark}{{\color{blue}\Large\ding{51}}}%
% \newcommand{\xmark}{{\color{red}\Large\ding{55}}}%
\title{Kỳ vọng 2} 
 \author{Toán Chuyên Đề}   
\institute[HUST]{HUST}
 
\maketitle  

\begin{frame}{Tài liệu tham khảo}
  \begin{itemize}
  \item Eric Lehman, F Thomson Leighton \& Albert R Meyer,
    \textit{Mathematics for Computer Science}, 2013
    \href{https://www.seas.harvard.edu/courses/cs20/MIT6_042Notes.pdf}{\color{blue}(Miễn
    phí)}
  \item Michael Mitzenmacher và Eli Upfal, \textit{Probability and Computing}, 2005
  \item  Nguyễn Tiến Dũng và Đỗ Đức Thái, \textit{Nhập Môn Hiện Đại Xác Suất \& Thống Kê}.
%  \item Phan .Đ. Diệu, \textit{Logic toán \& cơ sở toán học}. (2003)  
  \end{itemize}
\end{frame}
 
\begin{frame}
	\begin{thrm}
		Với mọi biến ngẫu nhiên $R_1, R_2, \dots, R_n$,
		\[
			\ex[R_1 + R_2 +\cdots + R_n] = \ex[R_1] + \ex[R_2] + \cdots + \ex[R_n].
		\]
	\end{thrm}
\end{frame}
%--- Next Frame ---%

\begin{frame}
  \frametitle{Kỳ vọng của phân phối nhị thức} 
  \begin{qstn}
    Tung $n$ đồng xu, mỗi đồng có xác suất xảy ra mặt ngửa là
    $p$. Kỳ vọng của số mặt ngửa bằng bao nhiêu?  
  \end{qstn}
\action<2->{Đặt 
$$
J = \text{ số mặt ngửa }
$$}
\action<3->{Vậy thì $J$ có phân phối nhị thức với tham số $n$ và $p$: 
\[
\pr[J=k] = \binom{n}{k}p^k(1-p)^{n-k}.
\]}
\end{frame}


\begin{frame}{Một lời giải khó}
\begin{align*}
\ex[J\ ] = \sum_{k=0}^n k\cdot \pr[J = k] &= \sum_{k=0}^nk\
                                        \binom{n}{k}p^k(1-p)^{n-k}\\
                                     &={\color{red}pn.}
\end{align*} 
\end{frame}

\begin{frame}
  \frametitle{Một lời giải dễ }
  
  \action<+->{Xét các biến ngẫu nhiên chỉ báo 
$$
J_i =
\begin{cases}
  1 &\text{ nếu đồng xu thứ $i$ ngửa }\\
 0 & \text{ ngược lại }
\end{cases}
$$}
\action<+->{Vậy thì số mặt ngửa là 
\[
J = J_1 + J_2 + \cdots + J_n.
\]}
\action<+->{Ta được 
\[
\ex[J\ ] = \action<+->{\sum_{i=1}^n \pr[J_i=1]} \action<+->{= pn.}
\]}
\end{frame}

\begin{frame}
  \frametitle{Chứng minh bằng phương pháp xác suất }
  Ta cũng vừa chứng minh đẳng thức phức tạp 
$$
\sum_{k=0}^nk\ \binom{n}{k}p^k(1-p)^{n-k} = pn.
$$
\end{frame}


\begin{frame}
  \begin{prblm}[Coupon Collector]
    \begin{itemize}
    \item Giả sử mỗi hộp bỏng ngô có kèm một chiếc tem.  
    \item Có $n$ loại
      tem khác nhau: xanh, đỏ, tím, vàng, cam... 
    \item Nếu bạn sưu tầm được  đủ mỗi  loại tem ít nhất một chiếc  thì bạn sẽ được
      nhận phần thưởng. 
    \item Giả sử chiếc tem trong mỗi hộp bỏng ngô được
      chọn ngẫu nhiên và theo phân phối đều.  
    \item Hỏi rằng bạn phải mua
      bao nhiêu hộp bỏng ngô thi mới được mỗi loại ít nhất một chiếc
      tem?
    \end{itemize}

  \end{prblm}
\end{frame}

\begin{frame}
  \frametitle{Ý tưởng }
  \begin{itemize}
  \item<+-> Giả sử ta có $5$ loại tem và ta có dãy tem sau đây
$$
\text{xanh}\quad \text{đỏ}\quad \text{đỏ} \quad \text{tím} \quad
\text{xanh} \quad \text{đỏ} \quad \text{vàng} \quad \text{xanh} \quad
\text{nâu}
$$
\item<+-> Ta phân hoạch thành các đoạn
$$
\underbrace{\text{xanh}}_{X_0}\quad \underbrace{\text{đỏ}}_{X_1}\quad
\underbrace{\text{đỏ} \quad \text{tím}}_{X_2} \quad
\underbrace{\text{xanh} \quad\text{đỏ}\quad \text{vàng}}_{X_3} \quad
\underbrace{\text{xanh} \quad \text{nâu}}_{X_4}
$$
theo quy tắc \alert{kết thúc một đoạn ngay khi ta gặp loại tem mới}.
\end{itemize}
\end{frame}

\begin{frame}{Lời giải bài toán Coupon Collector}
  \begin{itemize}
  \item<+-> Xét $X_k$ là độ dài của đoạn thứ $k$.
  \item<+-> Tổng số hộp bỏng ngô ta phải mua để có đủ $n$ loại tem là
    $$
    T = X_0 + X_1 + X_2 + \cdots + X_{n-1}.
    $$ 
  \end{itemize}
\end{frame}

\begin{frame}
  \frametitle{Kỳ vọng của $X_k$}
  \begin{itemize}
  \item<+-> Bắt đầu đoạn thứ $k$ ta có $k$ loại tem khác nhau, và đoạn thứ
    $k$ kết thúc khi ta có loại tem mới. 
  \item<+-> Mỗi hộp bỏng ngô  có
    thể kèm  một kiểu tem mới với xác suất là \action<+->{$$\frac{n-k}{n}.$$}
  \item<+-> Vậy thì 
    $$
    \ex[X_k] = \action<+->{ \frac{n}{n-k}.}
    $$
Tại sao?
  \end{itemize}
\end{frame}

\begin{frame}
  \frametitle{Lời giải bài toán Coupon Collector }
  \begin{align*}
    \ex[T] &= \action<+->{ \ex[X_0 + X_1 + \cdots + X_{n-1}] \\}
           \action<+->{&= \ex[X_0] + \ex[X_1] + \cdots + \ex[X_{n-1}] \\}
           \action<+->{&= \frac{n}{n-0} + \frac{n}{n-1} + \cdots + \frac{n}{3} +
             \frac{n}{2} + \frac{n}{1} \\}
           \action<+->{&= n\left(\frac{1}{n-0} + \frac{1}{n-1} + \cdots + \frac{1}{3} +
             \frac{1}{2} + \frac{1}{1}\right) \\}
           \action<+->{&\approx n\ln n.}
  \end{align*}
\end{frame}

\begin{frame}
  \frametitle{Ví dụ}
  Trung bình ta cần tung con xúc xắc khoảng
  $$
   6\times \left( 1 + 1/2 + 1/3 + 1/4 + 1/5 + 1/6\right )  \approx 14.7
  $$
lần để có đủ cả $6$ mặt xuất hiện.

\end{frame}
\begin{frame}
	\begin{thrm}
		Xét  không gian xác suất $\S$ và các sự kiện $A_1, A_2, \dots, A_n \subseteq \S$, vậy kỳ vọng của số lượng   sự kiện  xuất hiện bằng 
		\[
			 \pr[A_1] + \pr[A_2] + \cdots + \pr[A_n]
		\]
	\end{thrm}
      \end{frame}
      
\begin{frame}
          \begin{block}{Chứng minh.}
          \action<+->{Đặt $$T_i(w) =
          \begin{cases}
            1 &\text{ nếu } w \in A_i\\
            0 &\text{ ngược lại }
          \end{cases}$$
tức $T_i = 1$ nếu và chỉ nếu $A_i$ xảy ra;} \action<+->{ và  đặt 
$$T= T_1 + T_2 + \cdots + T_n.
$$}
\action<+->{Ta có 
\vspace{-0.5cm}
\begin{align*}
  \ex[T] &= \action<+->{\sum_{i=1}^n\ex[T_i ]   \\}
         \action<+->{&= \sum_{i=1}^n\pr[T_i =1]\\}
         \action<+->{&=\sum_{i=1}^n\pr[A_i].}
\end{align*}
}
\vspace{-0.7cm}
        \end{block}

\end{frame}
\begin{frame}
%--- Next Frame ---%
        \begin{xmpl}
          Tung $n$ đồng xu và đặt 
          \begin{align*}
            A_i &= \text{ sự kiện đồng xu thứ $i$ ngửa } \\
            T &= \text{ số mặt ngửa khi tung  $n$ đồng }
          \end{align*}
Ta có 
\begin{align*}
  \ex[T] &= \action<2->{\pr[A_1] + \pr[A_2] + \cdots + \pr[A_n] \\}
         \action<3->{&= 1/2 + 1/2 + \cdots + 1/2\\}
         \action<4->{&= n/2 }
\end{align*}
        \end{xmpl} 
\end{frame}

\begin{frame}
  \frametitle{Chứng minh khó hơn }
  \begin{align*}
    \ex[T] &= \sum_{i=1}^n i\cdot  \pr[T = i]&\\
           &= \sum_{i=1}^n i \cdot \binom{n}{i} 2^{-n}&  \text{(Giả sử các sự
                                                  kiện này 
             độc lập) }\\
           &= n/2
  \end{align*}
\end{frame}

\begin{frame}{Chứng minh bằng phương pháp xác suất}
  Như vậy ta đã chứng minh đẳng thức 
  $$
  \sum_{i=1}^n i \binom{n}{i}  = n\cdot 2^{n-1}
  $$
\end{frame}

\begin{frame}
  \begin{thrm}
    $$
    \underbrace{ \pr[T\geq 1]}_{\text{\color{blue!80}xác suất ít nhất một sự kiện $A_i$ xuất
        hiện }} \leq \ex[T]
    $$
  \end{thrm}
\action<2->{
  \begin{proof}
    \begin{align*}
      \ex[T] &= \sum_{i=1}^{\infty}\pr[T\geq 1] \\
             &\geq  \pr[T \geq 1].
    \end{align*}
  \end{proof}}
\end{frame}

\begin{frame}
  \begin{crllr}
    \begin{align*}
      \pr[T\geq 1] \leq \sum_{i=1}^{\infty} \pr[A_i].
    \end{align*}
  \end{crllr}
\end{frame}

\begin{frame}
  \begin{xmpl}
    \begin{itemize}
    \item Nếu $n = 1000$ và $\pr[A_i] = 1/100$, vậy thì
    $$
    \ex[T] = 10.
    $$
  \item Nếu với mọi $i,j$ ta có 
    $\pr[A_i\mid  A_j] = 1, 
    $
    vậy thì 
    $$
    \pr [T \geq 1] = 1/100 < 10.
    $$
  \end{itemize}
  \end{xmpl} 
\end{frame}

\begin{frame}
  \begin{thrm}[Luật Murphy] Xét các sự kiện độc lập $A_1, \dots,
    A_n$. Vậy thì 
    $$
    \pr[T = 0] \leq e^{-\ex[T]}.
    $$ 
  \end{thrm}
  \begin{proof}
\vspace{-0.5cm}
    \begin{align*}
      \pr[T=0] &= \pr[\overline{A_1} \cap \overline{A_2} \cap \cdots
                 \cap \overline{A_n}] &\\
               &= \prod_{i=1}^n \pr[\overline{A_i}] &\text{(do độc lập)}\\
               &= \prod_{i=1}^n (1-\pr[A_i]) &\text{(\text{do
                                               $1-x \leq e^{-x}$)}}\\
               &\leq \prod_{i=1}^n e^{-\pr[A_i]} = e^{-\sum_{i=1}^n
                 \pr[A_i] }
               = e^{-\ex[T]}. 
    \end{align*}

\vspace{-0.9cm}
  \end{proof}
\end{frame}

\begin{frame}
  \begin{crllr}
    Nếu mong đợi  ít nhất $10$  sự kiện độc lập xuất hiện, vậy thì xác
    suất \textbf{không} có sự kiện nào xuất hiện  $\leq e^{-10} < 1/22,000$. 
  \end{crllr}
\end{frame}

\begin{frame}
  \begin{thrm}[Luật tích]
    Với mọi cặp biến ngẫu nhiên \defi{độc lập} $R_1, R_2$
    $$
    \ex[R_1 \cdot R_2] = \ex[R_1]\cdot \ex[R_2].
    $$
  \end{thrm}

  \begin{xmpl}
    Tung hai con xúc xắc sáu mặt độc lập và đặt 
$$
R_i = \text{ kết quả của con xúc xắc } i 
$$ 
Ta có 
    \begin{align*}
      \ex[R_1 \cdot R_2] &= \ex[R_1] \cdot \ex[R_2] \\
                        &= 7/2 \cdot 7/2 = 49/4.
    \end{align*}
  \end{xmpl}
\end{frame}

\begin{frame}
  \frametitle{Điều kiện độc lập là cần thiết }
  \begin{xmpl}
    \begin{align*}
      \ex[R_1\cdot R_1] &= \ex[R_1^2] \\
                        &=\sum_{i=1}^{6} i^2 \pr[R_1 = i] \\
                        &= \frac{1}{16}\left( 1+4+9+ 16 + 25 + 36
                          \right)\\
                        &= 15\frac{1}{6} \not=
                          \left(3\frac{1}{2}\right)^2 = \ex[R_1]^2
    \end{align*}
  \end{xmpl}
\end{frame}

\begin{frame}
  \begin{crllr}
    Nếu $R_1, R_2, \dots, R_n$ là các biến ngẫu nhiên độc lập, vậy thì 
    $$
    \ex[R_1 \cdot R_2  \cdots R_n] = \ex[R_1]\cdot \ex[R_2] \cdots \ex[R_n].
    $$
  \end{crllr}
\end{frame}

\begin{frame}
  Chứng minh rằng: Với mọi hằng số $a,b$ và mọi biến ngẫu nhiên $R$, 
  $$
  \ex[a\cdot R  + b] = a \cdot  \ex[R] + b.
  $$ 
\end{frame} 

\begin{frame}
  \begin{qstn}
    Liệu đẳng thức  $$\ex\left[\frac{1}{R}\right] = \frac{1}{\ex[R]}$$ có đúng?
  \end{qstn}
\end{frame}

\begin{frame}
  \frametitle{Nghịch lý RISC về   kích thước mã nguồn}
  \begin{center}
    \begin{tabular}{lccc}
      Benchmark& RISC &CISC &CISC/RISC \\
      \hline 
      E-string search  &150 &120 &0.8 \\
      F-bit test       &120 &180 &1.5\\
      Ackerman         &150 &300 &2.0\\
      Rec Sort         &2800 &1400 &0.5 \\
\hline
                       &3220 &2000 
 
    \end{tabular}
  \end{center}
    \begin{align*}
      \text{Trung bình }[\text{ CISC/RISC }]&=\frac{1}{4} \times \left(  0.8 +
                                 1.5 +   2.0 +   0.5 \right)\\
      &= 1.2
    \end{align*}

  \begin{block}{Kết luận}
    ``Các chương trình CISC là dài hơn 20\% về trung bình.'' 
  \end{block}
\end{frame}

\begin{frame}
  \frametitle{Nghịch lý RISC về   kích thước mã nguồn 2}
  \begin{center}
    \begin{tabular}{lccc}
      Benchmark & RISC &CISC &RISC/CISC \\
      \hline 
      E-string search  &150 &120 &1.25 \\
      F-bit test       &120 &180 &0.67\\
      Ackerman         &150 &300 &0.5\\
      Rec Sort         &2800 &1400 &2.0 \\
\hline
      Trung bình                  & & &1.1
 
    \end{tabular}
  \end{center}

  \begin{block}{Kết luận}
    ``Các chương trình RISC là dài hơn 10\% về trung bình.'' 
  \end{block}
\end{frame}

\begin{frame}
  \frametitle{Tại sao?}
  \begin{itemize}
  \item Xét biến ngẫu nhiên $R$ là độ dài của chương trình viết bằng  RISC, 
  \item và biến ngẫu nhiên  $C$ là độ dài của chương trình viết bằng
    CISC.
  \item  Tính toán  
    $$
    \ex[C/R] = 1.2 \qquad \text{\cmark}
    $$
  \item Kết luận  ``chương trình CISC là dài hơn 20\% về trung bình''
    có nghĩa rằng 
    $$
    \ex[C] = 1.2\  \ex[R]\qquad  \text{\xmark}
    $$

  \end{itemize}
\end{frame}

\begin{frame}
  \frametitle{Một kết luận hợp lý }
  \begin{align*}
    \ex[R] &= \sum_{i \in Range(R)} i\cdot \pr[R = i]\\
           &= \frac{150}{4} + \frac{120}{4}
             +\frac{150}{4} + \frac{2800}{4}= 805.
  \end{align*}
  \begin{align*}
    \ex[C] &= \sum_{i \in Range(C)} i\cdot \pr[C = i]\\
           &= \frac{120}{4} + \frac{180}{4} +\frac{300}{4}
             +\frac{1400}{4} = 500.
  \end{align*}
Vậy $\ex[R]/\ex[C] = 1.61$. 
\begin{block}{Kết luận}
    ``Chương trình RISC là dài hơn 61\% về trung bình.''   
\end{block}
\end{frame}
\end{document}



%%% Local Variables:
%%% mode: latex
%%% TeX-master: t
%%% End:
