\documentclass[mathserif
%, handout
]{beamer}
 
  \useoutertheme{wuerzburg}
  \useinnertheme[outline]{chamfered}
 \usecolortheme{shark}
 \definecolor{MyBackground}{RGB}{243,246,249}
 \setbeamercolor{background canvas}{bg=MyBackground}

 %\usecolortheme[snowy]{owl}
 %\usecolortheme{owl}
 
 %\usetheme{Warsaw}
 %\usecolortheme{spruce}
\usepackage{rotating}
\usepackage[]{algorithm2e}
\usepackage{color, colortbl}
%\usepackage{default}
\usepackage{fontspec}
\usepackage{polyglossia} 
\setmainlanguage{vietnamese}
%\setdefaultlanguage{vietnamese} 
%\setmainfont{Palatino}
 \usepackage{wasysym}
\usepackage{pifont}% http://ctan.org/pkg/pifont

\usepackage{multicol}
\usepackage{sidecap}

\usepackage{hyperref}

\usepackage{pgf}    
\usepackage{tikz}
\usetikzlibrary{arrows,automata,decorations.pathmorphing,backgrounds,positioning,fit}
\usepackage{array}
\usepackage{listings}

\usepackage{enumerate}
%\usepackage{amsmath,mathtools}
%		\usepackage{fink}

\usepackage{amsmath,amsthm, amssymb}
\usepackage{microtype}
\usetikzlibrary{arrows,automata}
\usetikzlibrary{decorations.pathmorphing}


\usetikzlibrary{calc}

 

\usetikzlibrary{trees}
\usepackage{listings}

\setbeamertemplate{footline}[frame number]
\setbeamertemplate{navigation symbols}{}%remove navigation symbols

%\usepackage{listingsutf8}

%\setbeameroption{show notes on second screen=right}


  
%\newtheorem{Lemme}{Bổ đề} 
%\newtheorem*{LI}{Lemme d'itération infinie}

%\newtheorem{Proposition}{Mệnh đề}[section]
%\newtheorem{Theorem}{Định lý}[section]
%\newtheorem{Corollaire}{Hệ quả}[section]
%\newtheorem*{Conjecture}{Giả thuyết}
% \newtheorem*{Probleme}{Bài toán}
% \newtheorem*{Fait}{Fait}
% 
% 
% \theoremstyle{definition} \newtheorem{Definition}{Định nghĩa}
% \theoremstyle{definition} \newtheorem{example}{Ví dụ}
% \theoremstyle{remark} \newtheorem*{Remarque}{Chú ý}


\usetikzlibrary{arrows,automata}



\usetikzlibrary{trees}


% \newcommand{\mnvect}[2]
% {
%   \begin{bmatrix}	#1\\#2
%   \end{bmatrix}
% }

\definecolor{olive}{rgb}{0.3, 0.4, .1}
\definecolor{fore}{RGB}{249,242,215}
\definecolor{back}{RGB}{51,51,51}
\definecolor{title}{RGB}{255,0,90}
\definecolor{dgreen}{rgb}{0.,0.7,0. }
\definecolor{gold}{rgb}{1.,0.84,0.}
\definecolor{JungleGreen}{cmyk}{0.99,0,0.52,0}
\definecolor{BlueGreen}{cmyk}{0.85,0,0.37,0}
\definecolor{RawSienna}{cmyk}{0,0.72,1,0.45}
\definecolor{Magenta}{cmyk}{0,1,0,0} 



%\include{./myDefs}

%\setlength{\topmargin}{0cm} \setlength{\oddsidemargin}{0cm}
%\setlength{\evensidemargin}{0cm} \setlength{\textwidth}{17truecm}
%\setlength{\textheight}{21.0truecm}


%\parindent = 3 pt
%\parskip = 12 pt

%\newtheorem*{LI}{Lemme d'itération infinie}



\newtheorem{prprt}{Propriété}
\newtheorem{prpstn}{Mệnh đề}
\newtheorem{thrm}{Định lý}
\newtheorem{lmm}{Bổ đề}
\newtheorem{rl}{Luật}

\newtheorem{crllr}{Hệ quả}
\newtheorem{clm}{Khẳng định}
\newtheorem{nt}{Notation}
 
\newtheorem*{cnjctr}{Giả thuyết}

\newtheorem{fct}{Fait}
%\newtheorem{xmpl}{Exemple}
\newtheorem{rmrk}{Nhận xét}

\theoremstyle{example}
\newtheorem{xmpl}{Ví dụ}
\newtheorem{xrcs}{Bài tập}
  \newtheorem{dfntn}{Định nghĩa}
  \newtheorem{qstn}{Câu hỏi}
\newtheorem{prblm}{Bài toán}  
   \newtheorem{sol}{Lời giải}
  
%  \newtheorem{rmrk}{Định nghĩa}
  

% \declaretheorem[name=Problème]{prblm}
% \declaretheorem[name=Question, style=remark, numbered=no]{qstn}

% \declaretheorem[name=Théorème, numberwithin=section]{thrm}
% \declaretheorem[name=Lemme, sibling=thrm]{lmm}
% \declaretheorem[ name=Propriété, sibling=thrm]{prprt}
% \declaretheorem[ name=Proposition, sibling=thrm]{prpstn}
% \declaretheorem[name=Corollaire, sibling=thrm]{crllr}
% \declaretheorem[name=Fait, sibling=thrm]{fct}
% \declaretheorem[name=Notation, sibling=thrm]{nt}


% \declaretheorem[style=definition, name=Définition, sibling=thrm]{dfntn}

% %\theoremstyle{definition} \newtheorem{dfntn}{Définition}[section]

% \renewcommand\thmcontinues[1]{reprise de p.\,\pageref{#1}}

% \declaretheorem[style=remark, name=Exemple%, numberwithin=section
% ]{xmpl}

% \declaretheorem[style=remark, name=Remarque, numbered=no]{rmrk}

% %\declaretheorem[style=definition,numberwithin=chapter,name = Exemple]{xmpl}

% %\theoremstyle{remark} \newtheorem{xmpl}{Exemple}[chapter]

% %\theoremstyle{remark} \newtheorem*{rmrk}{Remarque}



\newtheorem{cs}{Cas}


\def\mclose{\texttt{close}}
\def\mopen{\texttt{open}}

\def\mmclose{\texttt{\scriptsize close}}
\def\mmopen{\texttt{\scriptsize open}}



% \newcommand{\mvect}[2]
% {
% \bigl[ \begin{smallmatrix}
% #1\\ #2
% \end{smallmatrix} \bigr]
% }

% \newcommand{\mnvect}[2]
% {
%   \begin{bmatrix}	#1\\#2
%   \end{bmatrix}
% }

% % \newcommand{\mnvect}[2]
% % {
% % #1/#2
% %   % \begin{bmatrix}	#1\\#2
% %   % \end{bmatrix}
% % }

% \newcommand{\XMPL}[3]
% {
%   \begin{xmpl}
%     Soient $L=\{#1\}$ et $\Sigma=\{#2\}$. On peut vérifier que $L$ est \orl\ avec le
%     relateur de base $#3$.
%   \end{xmpl}
% }

% \newcommand{\XMP}[4]
% {
%   \begin{xmpl}[#4]
%     Soient $L=\{#1\}$ et $\Sigma=\{#2\}$. On peut vérifier que $L$ est \orl\ avec le
%     relateur de base $#3$.
%   \end{xmpl}
% }

% \newcommand{\Pui}[2]
% {
%   #1^{\leq #2}
% }


% % \newcommand{\XMPL1}[4]
% % {
% %   \begin{xmpl}
% %     Soient $L=\{#1\}$ et $\Sigma=\{#2\}$. Il est clair que $L$ est \orl\ avec le
% %     relateur de base $#3$. $L^\omega$ est un 
% %   \end{xmpl}
% % }

% \def\vvs{\vspace{11pt}}
% \def\nni{\noindent}


% \newcommand{\cas}[1]
% {
% \vvs\nni
% \textbf{Cas #1 :}
% }



% \newcommand{\souscas}[1]
% {
% \vvs\nni
% \textbf{Sous-cas #1 :}
% }

% \def\pcom{paire de mots incompatibles}
% \def\wpcom{paire de mots $\infty$-incompatible}

% \def\upcom{une paire de mots incompatibles}
% \def\uwpcom{une paire de mots $\infty$-incompatibles}
% \def\comp{\asymp}

% \def\wg{code générateur}

%  \def\gc{code générateur}

% \def\gcx{codes générateurs}
% \def\Gcx{Codes générateurs }
% \def\ugc{un code générateur}
% \def\Ugc{un Code générateur}

% \def\wgc{$\omega$-code générateur}
% \def\wgcx{$\omega$-codes générateurs}
% \def\wGcx{$\omega$-Codes générateurs }
% \def\wugc{un $\omega$-code générateur}
% \def\wUgc{un $\omega$-code générateur}

% \def\orl {langage à un relateur}

% \def\orlx {langages à un relateur}
% \def\Orlx {Langages à un relateur}
% \def\uorl {un langage à un relateur}


% \def\ugc{un code générateur}

% \def\cp{code préfixe}

% \def\iff{si et seulement si} 
% \def\w{\omega}

% \def\CODE{la proposition~\ref{c3prop23}, $L^\omega$ n'a pas de \gc}
% \def\NOCODE{$L^\omega$ n'a pas de \gc}


\def\vs{}
\def\ni{}


\def\trail{hành trình đơn}
\def\Trail{Hành trình đơn}

\def\ctrail{\trail\ đóng}
\def\Ctrail{\Trail\ đóng }

\def\walk{hành trình}
\def\Walk{Hành trình}

\def\cwalk{hành trình đóng}
\def\Cwalk{Hành trình đóng}

\def\path{đường đi}
\def\Path{Đường đi}
 
\def\conn{liên thông}
\def\Conn{Liên thông}

\def\Comp{Thành phần liên thông}
\def\comp{thành phần liên thông}

\def\Cuted{Cạnh cắt}
\def\cuted{cạnh cắt}

\def\Cutve{Đỉnh cắt}
\def\cutve{đỉnh cắt}

\def\Induced{Đồ thị con cảm sinh}
\def\induced{đồ thị con cảm sinh}

 
\def\iff{{\color{blue} nếu và chỉ nếu}}

\def\ideg{\text{indeg}}
\def\odeg{\text{outdeg}}

\def\pr{\mathrm{Pr}}
\def\ex{\mathrm{Ex}}
\def\S{\mathcal{S}}
\def\var{\mathrm{Var}}


 \newcommand{\defi}[1]{{\color{blue}{\textbf{\emph{#1}}}}}
\newcommand{\contradiction}{{\hbox{%
    \setbox0=\hbox{$\mkern-3mu\times\mkern-3mu$}%
    \setbox1=\hbox to0pt{\hss$\times$\hss}%
    \copy0\raisebox{0.5\wd0}{\copy1}\raisebox{-0.5\wd0}{\box1}\box0
}}}

\newcommand{\cmark}{{\color{blue}\Large\ding{51}}}%
\newcommand{\xmark}{{\color{red}\Large\ding{55}}}%

%\newcommand{\defi}[1]{{\color{blue}{\textbf{\emph{#1}}}}}


 \AtBeginSection[]  
 { 
   \begin{frame}[plain]{Nội dung} 
     \tableofcontents[currentsection,currentsubsection] 
   \end{frame} 
 }  



\begin{document}
% \tikzstyle{every picture}+=[remember picture]

% \tikzstyle{na} = [baseline=-.5ex]

\author{Trần Vĩnh Đức}
%\institute[HUST]{Hanoi University of Science and Technology}



% \newcommand{\cmark}{{\color{blue}\Large\ding{51}}}%
% \newcommand{\xmark}{{\color{red}\Large\ding{55}}}%
\title{Độ lệch}  
\author{Toán Chuyên Đề}   
\institute[HUST]{HUST} 

\maketitle    

\begin{frame}{Tài liệu tham khảo} 
  \begin{itemize}
  \item Eric Lehman, F Thomson Leighton \& Albert R Meyer,
    \textit{Mathematics for Computer Science}, 2013
    \href{https://www.seas.harvard.edu/courses/cs20/MIT6_042Notes.pdf}{\color{blue}(Miễn
      phí)}
  \item Michael Mitzenmacher và Eli Upfal, \textit{Probability and Computing}, 2005
  \item  Nguyễn Tiến Dũng và Đỗ Đức Thái, \textit{Nhập Môn Hiện Đại Xác Suất \& Thống Kê}.
    % \item Phan .Đ. Diệu, \textit{Logic toán \& cơ sở toán học}. (2003)  
  \end{itemize}
\end{frame}

\section{Phương sai}
\begin{frame}{Ví dụ}
  \begin{center}
    % \begin{minipage}{0.45\textwidth}
    \begin{block}{Trò chơi A}
      Bạn sẽ thắng $\$2$ với xác suất $2/3$ và thua $\$1$ với xác suất $1/3$.
    \end{block}
    % \end{minipage}
    % \begin{minipage}{0.5\textwidth}
    \begin{block}{Trò chơi B}
      Bạn sẽ thắng $\$1002$ với xác suất $2/3$ và thua $\$2001$ với xác suất $1/3$.		
    \end{block}
    % \end{minipage}		
  \end{center}
  \action<2->{Bạn nên chơi trò chơi nào?} \action<3->{ Kỳ vọng lãi thu được từ mỗi trò chơi là bao nhiêu?}   
\end{frame}

\begin{frame}{Kỳ vọng lãi}
  \begin{align*}
    \action<+->{	\ex [A] &= 2 \cdot \frac{2}{3} + (-1) \cdot \frac{1}{3} = 1   \\}
    \action<+->{	\ex [B] &= 1002 \cdot \frac{2}{3} + (-2001) \cdot \frac{1}{3} = 1 }
  \end{align*}
  \action<+->{\begin{qstn}
      Trò chơi nào rủi ro hơn?	
    \end{qstn}}

\end{frame}

\begin{frame}
  \begin{dfntn}
    \defi{Phương sai} của biến ngẫu nhiên $R$ là 
    \[	
      \var [R] = \ex[(R - \ex[R])^2]
    \]
  \end{dfntn}
  Nói cách khác, phương sai là  trung bình của  bình phương độ lệch so với trung bình. 
\end{frame}
% --- Next Frame ---%

\begin{frame}{Trò chơi A}
  \begin{align*}
    \action<+->{A - \ex[A] &= \begin{cases}
        1 &\text{ với xác suất } 2/3\\
        -2 &\text{ với xác suất } 1/3\\
      \end{cases}\\}
    \action<+->{(A - \ex[A])^2 &= \begin{cases}
        1 &\text{ với xác suất } 2/3\\
        4 &\text{ với xác suất } 1/3\\
      \end{cases} \\}
    \action<+-> {\ex[(A - \ex[A])^2] &= 1 \cdot \frac{2}{3} + 4 \cdot \frac{1}{3} \\}
    \action<+->{\var[A] &= 2. }
  \end{align*}
\end{frame}
% --- Next Frame ---%

\begin{frame}{Trò chơi B}
  \begin{align*}
    \action<+->{B - \ex[B] &= \begin{cases}
        1001 &\text{ với xác suất } 2/3\\
        -2002 &\text{ với xác suất } 1/3\\
      \end{cases}\\}
    \action<+->{(B - \ex[B])^2 &= \begin{cases}
        1, 002,001 &\text{ với xác suất } 2/3\\
        4, 008, 004 &\text{ với xác suất } 1/3\\
      \end{cases} \\}
    \action<+->{\ex[(B - \ex[B])^2] &= 1,00,001 \cdot \frac{2}{3} + 4,008,004 \cdot \frac{1}{3} \\}
    \action<+->{\var[B] &= 2,004,002. }
  \end{align*}
\end{frame}
% --- Next Frame ---%

\begin{frame}{Trò chơi nào rủi ro hơn?}
  \begin{center}
    % \begin{minipage}{0.45\textwidth}
    \action<+->{		\begin{block}{Trò chơi A}
        \[
          \var[A] = 2
        \]
        Lãi suất thường gần với giá trị trung bình $\$1$.
      \end{block}}
    % \end{minipage}
    % \begin{minipage}{0.5\textwidth}
    \action<+->{		\begin{block}{Trò chơi B}
        \[
          \var[B] = 2,0004,002
        \]
        Lãi suất lệch rất xa so với giá trị trung bình là $\$1$.
      \end{block}}
    % \end{minipage}		
  \end{center}
  \action<+->{Phương sai cao thường gắn với rủi ro nhiều. Ví dụ, trong 10 lần chơi trò chơi A, ta có lãi trung bình $\$10$ nhưng cũng có thể mất $\$10$.	Còn với trò chơi B thì sao?}
\end{frame}
% --- Next Frame ---%
\begin{frame}{"Đơn vị" của phương sai}
  \begin{itemize}
  \item<+-> Biến ngẫu nhiên và phương sai không cùng "đơn vị".
    \[	
      \var [R] = \ex[(R - \ex[R])^2]
    \]
  \item<+-> Ví dụ, nếu đơn vị của biến ngẫu nhiên là $\$$, vậy thì đơn vị của phương sai là $\$^2$.
  \item<+-> Độ lệch chuẩn tương tự như phương sai nhưng cùng "đơn vị" với biến ngẫu nhiên.   
    
  \end{itemize}
\end{frame}
% --- Next Frame ---%

\begin{frame}
  \begin{dfntn}
    \defi{Độ lệch chuẩn} $\sigma_R$ của biến ngẫu nhiên $R$ là căn bậc hai của phương sai:
    \[
      \sigma_R = \sqrt{\var[R]} = \sqrt{\ex[(R - \ex[R])^2]}.
    \] 
  \end{dfntn}
  \action<2->{	\begin{xmpl}
      Độ lệch chuẩn của biến ngẫu nhiên lãi trong trò chơi A và B là 
      \begin{align*}
        \action<3->{\sigma_A &= \sqrt{\var[A]} = \sqrt{2} \approx 1.14, \\}
        \action<4->{\sigma_B &= \sqrt{\var[B]} = \sqrt{2,004,002} \approx 1416.}
      \end{align*} 
    \end{xmpl}}
  \note{Biến ngẫu nhiên $B$ thực sự lệch so với trung bình bởi hoặc dương $1001$ hoặc âm $2002$; vậy nên độ lệch chuẩn $1416$ mô tả khá hợp lý tình huống này.}
\end{frame}
% --- Next Frame ---%

\begin{frame}{Công thức khác cho phương sai}
  \begin{lmm}
    Với mọi biến ngẫu nhiên $R$,
    \[
      \var[R] = \ex[R^2] - (\ex[R])^2.
    \]
  \end{lmm}
  \action<2->{	\begin{xmpl} Với trò chơi A
      \begin{align*}
        \action<3->{\ex[A] &= 2 \cdot \frac{2}{3} + (-1)\cdot \frac{1}{3} = 1 \\}
        \action<4->{\ex[A^2] &= 2^2 \cdot \frac{2}{3} + (-1)^2\cdot \frac{1}{3} = 3\\}
        \action<5->{\var[A] &= \ex[A^2] - (\ex[A])^2 = 3 - 1^2 = 2.}
      \end{align*}
    \end{xmpl}}
\end{frame}
% --- Next Frame ---%
\begin{frame}{Bài tập}
  Hãy chứng minh bổ đề trước.
\end{frame}
\begin{frame}{Phương sai của biến ngẫu nhiên chỉ báo}
  \begin{lmm}
    Xét $B$ là biến ngẫu nhiên chỉ báo với $\pr[B=1] = p$. Vậy thì 
    \[
      \var[B] = p(1-p).
    \]
  \end{lmm}
  \action<2->{	\begin{xrcs} 
      Hãy chứng minh bổ đề trên.
    \end{xrcs}}
\end{frame}
% --- Next Frame ---%
% --- Next Frame ---%
\begin{frame}{Số bước trung bình dẫn đến lỗi }
  \begin{itemize}
  \item<+-> Hệ thống lỗi ở mỗi bước với xác suất $p$.
    % \item<+-> Vậy trung bình  $1/p$ bước sẽ xảy ra lỗi đầu tiên.
  \item<+-> Xét $C$ là số bước  để có lỗi đầu tiên  xuất hiện (kể cả bước lỗi). Vậy 
    \[
      \ex[C] = \action<+->{1/p.}
    \]  
  \item<+-> Phương sai của $C$ bằng bao nhiêu? 
  \end{itemize}
\end{frame}
% --- Next Frame ---%

\begin{frame}
  \begin{align*}
    \ex[C^2] &= \action<+->{\overbrace{1^2\cdot p}^{\text{lỗi ngay bước đầu tiên}}\ +\  \overbrace{\ex [(C+1)^2]\cdot (1-p)}^{\text{hoặc không}} \\}
    \action<+->{             &= p + \ex[C^2]\cdot (1-p) + 2\cdot \ex[C]\cdot (1-p) + (1-p)\\  }
    % &\text{Hệ thống có thể lỗi ngay bước đầu tiên hoặc không}\\
    \action<+->{&= 1 + \ex[C^2]\cdot (1-p) + 2\cdot \left(\frac{1-p}{p}\right)}
  \end{align*}
  \action<+->{	Ta được 
    \[
      p\cdot \ex[C^2] = \frac{2-p}{p}.
    \]}
  \action<+->{	và được 
    \[
      \ex[C^2] = \frac{2-p}{p^2}.
    \]}
\end{frame}
% --- Next Frame ---%

\begin{frame}{Bài tập}
  Hãy tính tiếp $\var[C]$.
\end{frame}
% --- Next Frame ---%

\begin{frame}{Bài tập: Biến ngẫu nhiên đều}
  Với biến ngẫu nhiên đều $R$ trên $\{1,2,3, \dots, n\}$, phương sai của $R$ bằng bao nhiêu?
  
\end{frame}
% --- Next Frame ---%
\begin{frame}
  \begin{thrm}
    Nếu $R_1, R_2$ là hai biến ngẫu nhiên \alert{độc lập}, vậy thì 
    \[
      \var[R_1 + R_2] = \var[R_1] + \var[R_2].
    \] 	
  \end{thrm}
  \action<2->{Bài tập: Hãy chứng minh định lý trên.}
  
\end{frame}
% --- Next Frame ---%

\begin{frame}%{Phân phối nhị thức}
  \begin{lmm}
    Xét $J$ là biến ngẫu nhiên theo phân bố nhị thức với tham số $n$ và~$p$, vậy thì 
    \[
      \var[J] = n p(1-p).
    \] 		
  \end{lmm}\pause
  \begin{block}{Chứng minh}
    \action<+->{Xem $J$ như số mặt "ngửa" khi tung $n$ đồng xu độc lập, mỗi đồng có xác suất xuất hiện mặt ngửa là $p$.} \action<+->{Đặt 
      \[
        J_i = 1\quad  \Leftrightarrow\quad \text{ đồng thứ $i$ ngửa}  
      \]
      \vspace{-0.5cm}}
    
    \action<+->{		Vậy thì 
      \vspace{-0.5cm}
      \begin{align*}
        \var[J] &= \action<+->{\var[J_1 + J_2 + \cdots + J_n]\\}
        \action<+->{&= \var[J_1] + \var[J_2] + \cdots + \var[J_n]\\}
        \action<+->{&= n p(1-p).}
      \end{align*}}
    \vspace{-0.5cm}
    
  \end{block}
\end{frame}
% --- Next Frame ---%

\section{Định lý Markov}
\begin{frame}{Ví dụ}
  \begin{itemize}
  \item<+-> \alert{I}ntelligent \alert{Q}uotients trung bình của mọi người là $100$.
  \item<+-> Vậy nhiều nhất chỉ $1/3$ dân số có IQ lớn hơn $300$. Tại sao?
  \item<+-> Suy ra,  xác suất một người ngẫu nhiên có IQ lớn hơn $300$ là $\leq 1/3$.  
  \end{itemize}
\end{frame}

\begin{frame}
  \begin{thrm}[Markov]
    Nếu $R$ là biến ngẫu nhiên \alert{không âm}, vậy thì với mọi $x > 0$,
    \[
      \pr[R \geq x] \leq \frac{\ex[R]}{x}.
    \] 
  \end{thrm}
\end{frame}
% --- Next Frame ---%

\begin{frame}{The Chinese Appetizer Problem}
  \begin{itemize}
  \item Có $n$ người ngồi ăn quanh một mâm tròn. 
  \item<+-> Mỗi người có một món khai vị trước mặt. Giả sử các món khai vị này khác nhau.
  \item Lợi dụng lúc mọi người mải nói chuyện, ai đó đã quay mâm một cách ngẫu nhiên để mỗi người nhận được ngẫu nhiên một món khai vị.
  \item Hãy tính xác suất để cả $n$ người đều nhận lại được đúng món khai vị của mình. 
  \end{itemize}
\end{frame}
% --- Next Frame ---%

\begin{frame}{The Chinese Appetizer Problem}
  \begin{itemize}
  \item<+->  Dễ thấy,  mỗi người nhận lại được món khai vị ban đầu của mình với xác suất $1/n$.
  \item<+->  Kỳ vọng của số người $R$ nhận đúng món khai vị của mình là 
    \[
      \ex[R] = n\cdot \frac{1}{n}.
    \]  
    
  \item<+-> Theo định lý Markov,
    \[
      \pr[R =n] = \pr[R \geq n] \leq \frac{\ex[R]}{n} = \frac{1}{n}.
    \] 
  \end{itemize}

  \action<+->{Trong trường hợp này định lý Markov cho ta kết quả chính xác.}
\end{frame}
% --- Next Frame ---%

\begin{frame}{Bài tập}
  Hãy chứng minh định lý Markov.
\end{frame}

\begin{frame}{Giả thiết $R$ không âm là quan trọng}
  \action<+->{Xét biến ngẫu nhiên $R$ với 
    \[
      \pr[R = 1000] = 1/2\quad  \text{ và }\quad \pr[R=-1000] = 1/2.
    \]}
  \action<+->{Vậy thì 
    \begin{align*}
      \ex[R] &= 0. \\
      \pr[R \geq 1000] &= 1/2 \not= \frac{\ex[R]}{1000} = 0. 
    \end{align*}}
\end{frame}
% --- Next Frame ---%
\begin{frame}{Biến ngẫu nhiên bị chặn}
  \begin{itemize}
  \item<+-> Giả sử IQ trung bình của sinh viên Bách Khoa là $150$.
  \item<+-> Xác suất một sinh viên Bách Khoa có IQ hơn $200$ khoảng bao nhiêu?
    \action<+->{\[
        \pr[B \geq 200] \leq \frac{\ex[B]}{200} = \frac{150}{200} = \frac{3}{4}.  
      \]}
  \item<+-> Biết thêm rằng không có sinh viên nào có IQ nhỏ hơn $100$, vậy ước lượng trên có thể giảm xuống bằng bao nhiêu?
    
  \end{itemize}
\end{frame}

\begin{frame}
  Xét $T = B + 100$, ta được
  \[
    \pr[B \geq 200] = \pr[T \geq 100] \leq \frac{\ex[T]}{100} = \frac{50}{100} = \frac{1}{2}.
  \]
\end{frame}
% --- Next Frame ---%

\begin{frame}
  \begin{crllr}
    Nếu $R$ là biến ngẫu nhiên \alert{không âm}, vậy thì với mọi $c > 0$
    \[
      \pr\left[R \geq c\cdot \ex[R]\right] \leq 1/c.
    \]
  \end{crllr}
  \begin{proof}
    \action<2->{		Thay $x = c \cdot \ex[R]$ vào định lý Markov.}
  \end{proof}
\end{frame}
% --- Next Frame ---%

\begin{frame}
  \begin{thrm}
    Xét $R$ là biến ngẫu nhiên thỏa mãn  $R\leq u$. Vậy thì với mọi $x < u$,
    \[
      \pr[R \leq x] \leq \frac{u-\ex[R]}{u-x}.
    \]
  \end{thrm}
  \action<2->{Bài tập: Hãy chứng minh định lý trên.}
\end{frame}

\begin{frame}{Bài tập}
  \begin{itemize}
  \item<+-> Giả sử điểm thi giữa kỳ trung bình của lớp Toán Chuyên Đề là
    $7.5/10$. 
  \item<+-> Hãy ước lượng tỉ lệ sinh viên trong lớp có điểm nhỏ hơn
    hoặc bằng $5$.
  \end{itemize}
\end{frame}
% --- Next Frame ---%
\begin{frame}{Giải}
  \begin{align*}
    \action<+->{R &= \text{ điểm ngẫu nhiên của sinh viên } \\}
    \action<+->{\max{\text{ Điểm }} &= 10 = u \\}
    \action<+->{\ex[R] &= 7.5 \\}
    \action<+->{\pr[R \leq 5.0] &\leq \frac{10.0 - 7.5}{10.0 - 5.0} = \frac{2.5}{5.0} = \frac{1}{2}. }
  \end{align*}
  \action<+->{	Nói cách khác, chỉ nhiều nhất nửa lớp có điểm $\leq 5$.}
\end{frame}
% --- Next Frame ---%
% --- Next Frame ---%
\section{Định lý Chebyshev}

\begin{frame}
  \begin{lmm}
    Với mọi biến ngẫu nhiên $R$, $z \in \mathbb{R}$, và $x > 0$,
    \[
      \pr[|R| \geq x] \leq \frac{\ex[|R|^z]}{x^z}
    \]
  \end{lmm}
  \action<2->{	\begin{proof}
      Do 
      \[
        |R| \geq x \quad \Leftrightarrow\quad |R|^z \geq x^z 
      \]
      Áp dụng định lý Markov, ta suy ra bổ đề trên.
    \end{proof}}
\end{frame}
% --- Next Frame ---%

\begin{frame}
  \begin{thrm}[Chebyshev]
    Xét $R$ là một biến ngẫu nhiên và $x \in \mathbb{R}^+$. Vậy thì
    \[
      \pr[|R - \ex[R]|\geq x] \leq \frac{\var[R]}{x^2}.
    \]
  \end{thrm}
  \action<2->{Đây là một trường hợp riêng của bổ đề trước. Tại sao?}
\end{frame}

\begin{frame}
  \begin{crllr}
    Xét $R$ là biến ngẫu nhiên, và xét $c$ là một số thực dương 
    \[
      \pr[|R - \ex[R]|\geq c\cdot  \sigma_R] \leq \frac{1}{c^2}.
    \]
  \end{crllr}
  \action<2->{Bài tập: Hãy chứng minh hệ quả trên.}
\end{frame}
% --- Next Frame ---%

\begin{frame}
  \begin{xmpl}
    $R = $ IQ của một người ngẫu nhiên. Giả sử 
    \[
      R \geq 0,\quad  \ex[R] = 100,\quad \sigma_R = 15 
    \]
    Hãy ước lượng 
    \[
      \pr[R \geq 250].
    \]
  \end{xmpl}
\end{frame}
% --- Next Frame ---%
\begin{frame}{Ước lượng}
  \begin{itemize}
  \item<+-> Bởi định lý Markov
    \action<+->{\[
        \pr[R \geq 250 ] \leq \frac{\ex[R]}{250} = \frac{100}{250} = 0.4
      \]}
  \item<+-> Bởi định lý Chebyshev
    \begin{align*}
      \action<+->{\pr[R \geq 250] &= \pr[R - 100 \geq 150] \\}
      \action<+->{&= \pr[R - \ex[R] \geq 10\cdot \sigma_R] \\}
      \action<+->{	&\leq \pr[|R - \ex[R]| \geq 10 \cdot \sigma_R]\\}
      \action<+->{	&\leq \frac{1}{100}.}
    \end{align*}
  \end{itemize}
\end{frame}
% --- Next Frame ---%

\begin{frame}
  \begin{thrm}
    Với mọi biến ngẫu nhiên $R$ và mọi $c > 0$
    \[
      \pr[R - \ex[R] \geq c\cdot \sigma_R] \leq \frac{1}{c^2 + 1}
    \]
    và 
    \[
      \pr[R - \ex[R] \leq  -c\cdot \sigma_R] \leq \frac{1}{c^2 + 1}.
    \]
  \end{thrm}
\end{frame}
% --- Next Frame ---%
\begin{frame}{Trở lại với {\color{red}IQ}}

  \begin{xmpl}
    $R = $ IQ của một người ngẫu nhiên. Giả sử 
    \[
      R \geq 0,\quad  \ex[R] = 100,\quad \sigma_R = 15 
    \]
    Hãy ước lượng 
    \[
      \pr[R \geq 250].
    \]
  \end{xmpl}
\end{frame}

\begin{frame}{Ước lượng}
  \begin{align*}
    \action<+->{  \pr[R \geq 250] &= \pr[R - 100 \geq 150] \\}
    \action<+->{&= \pr[R - \ex[R] \geq 10 \cdot \sigma_R] \\}
    \action<+->{&\leq \frac{1}{10^2 + 1} = \frac{1}{101}}
  \end{align*}
\end{frame}
\section{Chặn của tổng các biến ngẫu nhiên}
\begin{frame}{Chặn Chernoff}
  \begin{minipage}[c]{0.6\textwidth}
    Tổng của rất nhiều biến ngẫu nhiên có giá trị nhỏ và độc lập có nhiều khả năng sẽ  không vượt quá trung bình của tổng. 
  \end{minipage}\qquad 
  \begin{minipage}[c]{0.3\textwidth}
    \begin{figure}[h]
      \centering
      \includegraphics[width=\textwidth]{Chernoff.pdf}
    \end{figure}
    
  \end{minipage}
\end{frame}
% --- Next Frame ---%
\begin{frame}
  \begin{thrm}[Chặn Chernoff]
    Xét $T_1,T_2, \dots T_n$ là các biến ngẫu nhiên độc lập thoả mãn $0\leq T_i \leq 1$ với mọi $i$. Xét 
    \[
      T = T_1 + T_2 + \cdots + T_n.
    \]
    Vậy thì với mọi $c\geq 1$,
    \[
      \pr[T \geq c\ \ex[T]] \leq e^{-k\ \ex[T]}
    \]
    trong đó $k = c\ \ln(c) - c +1$.
  \end{thrm}
\end{frame}
\begin{frame}
  \begin{xmpl}
    Tung $1000$ đồng xu độc lập. Hãy tính xác suất của số mặt ngửa vượt quá kỳ vọng ít nhất $20\%$.
  \end{xmpl}	


\end{frame}

\begin{frame}
  \begin{sol}
    \begin{itemize}
    \item<+-> Đặt $T_i$ là biến ngẫu nhiên chỉ báo cho sự kiện đồng xu thứ $i$ là ngửa.
    \item<+-> Vậy thì tổng số mặt ngửa là 
      \[
        T = T_1 + T_2 + \cdots + T_{1000}.
      \] 
      
    \item<+-> Cả hai điều kiện của Chernoff đều thoả mãn: Các biến $T_i$ độc lập và $T_i \in [0,1]$. 
    \item<+-> Theo định lý Chernoff
      \[
        \pr[T \geq c\ \ex[T]] \leq e^{-k\ \ex[T]}
      \] 
      với $c = 1.2$ và  $k = c\ \ln(c) - c + 1 = 0.0187\dots$.
    \end{itemize}    
  \end{sol}
\end{frame}
\begin{frame}
  \begin{align*}
    \pr[T \geq 1.2\ \ex[T]] &\leq e^{-k\ \ex[T]}\\
                            &= e^{-(0.0187\dots)\cdot 500}\\
                            &< 0.0000834
  \end{align*}
\end{frame}
% --- Next Frame ---%

\begin{frame}{Ảnh hưởng của số biến và độ lệch }
  Xác suất sẽ nhỏ hơn rất nhiều nếu số đồng xu tăng lên. 
  \begin{xmpl}
    Nếu tung $1$ triệu đồng xu, xác suất để số mặt
    ngửa vượt quá kỳ vọng ít nhất $20\%$ chỉ nhiều
    nhất là
    \[
      e^{-(0.0187)\cdot 500000} < e^{-9392}.
    \]
  \end{xmpl}\pause 

  
  Xác suất cũng sẽ nhỏ hơn rất nhiều nếu độ lệch tăng lên.
  \begin{xmpl}
    Tung $1000$ đồng xu, xác suất để số mặt ngửa vượt quá kỳ vọng ít
    nhất $30\%$ chỉ nhiều nhất là
    \[
      e^{-(0.0410)\cdot 500} < e^{-20.5}< 1/1,000,000,000.
    \]
  \end{xmpl}

\end{frame}
% --- Next Frame ---%

\begin{frame}{Trò chơi Pick-4}
  \begin{itemize}
  \item<+-> Bạn chọn một số bốn chữ số trong khoảng $0000$ đến $9999$.
  \item<+-> Nếu số bạn chọn là số ngẫu nhiên chương trình chọn, bạn sẽ được $\$5,000$.
  \item<+-> Xác suất thắng của bạn là $1/10,000$.
  \item<+-> Nếu có $10$ triệu người chơi, kỳ vọng số người thắng là $1000$. 
    
  \item<+-> Nỗi lo của công ty sổ xố: Số người thắng ít nhất là $2000$.
  \item<+-> Hãy tính xác suất để số người thắng ít nhất là $2000$.    
  \end{itemize}
\end{frame}
% --- Next Frame ---%

\begin{frame}[t]{Trò chơi Pick-4 (lời giải)}
  \begin{itemize}
    
  \item<+-> Đặt $T_i$ là biến chỉ số cho sự kiện người thứ $i$ thắng.
  \item<+-> Số người thắng là biến $	T = T_1 + T_2 + \cdots + T_{10,000,000}$.

  \item<+-> Vì số người thắng gấp $2$ lần kỳ vọng, ta chọn $c = 2$.
    
  \item<+-> Ta giả sử người chơi chọn số ngẫu nhiên đều và  độc lập. Vậy thì   Theo định lý Chernoff
    \begin{align*}
      k = c\ \ln(c) - c + 1 &= 0.386 \\
      \pr[T \geq 2000] &= \pr[T \geq 2\ \ex[T]] \\
                            &\leq e^{-k\ \ex[T]} \\
                            &=e^{-(0.386\dots)\cdot 1000}\\
                            &< e^{-386}
    \end{align*}  
    Vậy hầu như không bao giờ công ty sổ xố phải trả gấp đôi kỳ vọng.
  \end{itemize}
\end{frame}
% --- Next Frame ---%

\begin{frame}{Trò chơi Pick-4 (tiếp)}
  \begin{xrcs}
    Hãy tính xác suất để số người thắng  cao hơn $10\%$ so với kỳ vọng.
  \end{xrcs}
  \action<2->{\begin{align*}
                k = 1.1\ \ln(1.1) - 1.1 + 1 &= 0.00484 \\
                \pr[T \geq 1.1\ \ex[T]] &\leq e^{-k\ \ex[T]} \\
                                            &= e^{-(0.00484) \cdot 1000}\\
                                            &< 0.01
              \end{align*}}
          \end{frame}
          % --- Next Frame ---%

          \section{Ứng dụng: Bài toán cân bằng tải}
          \begin{frame}
            \begin{block}{Cân bằng tải}
              \begin{itemize}
              \item Hệ thống với $n$ công việc $B_1, B_2, \dots, B_n$ đến theo dòng.
              \item Công việc $B_i$ cần $L_i$ thời gian
                
              \item  Các công việc cần xử lý ngay
                lập tức trên $m$ máy $S_1, S_2, \dots, S_n$.
              \item Hãy tìm cách gán mỗi công việc cho mỗi máy để hệ thống đảm bảo cân bằng tải.
              \end{itemize}
            \end{block}\pause 

            \begin{block}{Phương pháp}
              Gán một cách ngẫu nhiên mỗi công việc đến cho một máy.  
            \end{block}
          \end{frame}

          \begin{frame}{Dữ liệu thực tế}
            \begin{itemize}
            \item Số công việc $n = 100,000$.
            \item Số lượng máy $m = 10$. 
            \item Đặt $$L = \sum_{j=1}^{n} L_j.$$
            \item Giả sử $L = 25,000$ giây.
              
            \item Vậy tải trung bình trên mỗi máy 
              \[
                \frac{L}{m} = \frac{25,000}{10} = 2500.
              \]  
            \end{itemize}
          \end{frame}
          % --- Next Frame ---%
          \begin{frame}{Phân tích}
            \begin{itemize}
            \item Đặt $R_{ij} $ là tải trên  máy $S_i$ từ công việc $B_j$. Tức là 
              $$
              R_{ij} = \begin{cases}
                L_j &\text{nếu máy $S_i$ được gán công việc $B_j$ }\\
                0 & \text{ngược lại.}
              \end{cases} 
              $$
            \item Vậy thì tải của máy $S_i$ là $$R_i =  R_{i1} + R_{i2} + \cdots + R_{in}.$$
            \item Ta được 
              \begin{align*}
		\ex[R_i] &= \sum_{j=1}^{n} \ex[R_{ij}] \\
		         &= \sum_{j=1}^n L_j/m\\
                         &= L/m.
              \end{align*}
              Đây là giá trị tối ưu cân bằng tải.
            \end{itemize}
          \end{frame}

          \begin{frame}{Phân tích 2: Tải của mỗi máy $R_i$}
            \begin{itemize}
            \item 	Giả sử các $0\leq R_{ij}\leq 1$, theo định lý Chernoff,
              \[
		\pr[R_i \geq c\ L/m] \leq e^{-k\ L/m}
              \]
              với $k = c \ln(c) - c + 1$.
              
              
            \item Với $c = 1.1$, ta được $k = 0.0048$,
            \item và với  $L = 25,000$ ta được
              \[
		\pr[R_i \geq 1.1\times L/m] \leq e^{-0.0048\times 2500} \leq 1/160,000.
              \]
            \end{itemize}
          \end{frame}
          % --- Next Frame ---%

          \begin{frame}{Phân tích 3: Máy phải chịu tải nhiều nhất}
            \begin{align*}
              &\pr[\text{ máy chịu tải nhiều nhất } \geq c\ L/m\ ] \\
              &= \pr[\ (R_1 \geq c\ L/m)\  \cup\ (R_2 \geq c\ L/m)\  \cup\ \cdots\ \cup\ (R_m \geq c\ L/m)\  ]\\
              &\leq \sum_{i=1}^m \pr[R_i \geq c\ L/m] \\
              &\leq \frac{m}{160,000} = \frac{1}{16,000}. 
            \end{align*}
          \end{frame}
          % --- Next Frame ---%
        \end{document}



        %%% Local Variables:
        %%% mode: latex
        %%% TeX-engine: xetex
        %%% TeX-source-correlate-mode: t
        %%% TeX-master: t
        %%% End:
